\newpage
\subsection{Rare Scenarios Tests}

\begin{samepage}
\subsection*{62. Lightning Surge Test}

\subparagraph{Test Goals and Purpose}
\begin{itemize}
    \item Assess hardware resilience to sudden electrical surges or transient spikes.
    \item Verify lock recovers to normal operation after surge exposure.
\end{itemize}

\subparagraph{How We Test It}
\begin{itemize}
    \item Use a surge generator to apply a 1-kV spike to the 12V input for 1 microsecond.
    \item Monitor the lock’s immediate and long-term response.
\end{itemize}

\subparagraph{Expectations of Test}
\begin{center}
\begin{tabular}{|c|p{10cm}|}
  \hline
  \textbf{Result} & \textbf{Conditions} \\
  \hline
  \textbf{PASS} &
    \begin{minipage}[t]{\linewidth}
    \begin{itemize}
      \item No permanent damage to circuitry or lock mechanism.
      \item Lock operates normally after the surge event.
      \item No residual performance degradation observed.\\
    \end{itemize}
    \end{minipage} \\
  \hline
  \textbf{FAIL} & Any permanent malfunction, damage, or inability to resume normal operation after surge exposure. \\
  \hline
\end{tabular}
\end{center}
\end{samepage}

\newpage
\begin{samepage}
\subsection*{63. Seismic Vibration Test}

\subparagraph{Test Goals and Purpose}
\begin{itemize}
    \item Ensure the lock’s mechanical and electronic components remain secure and functional during prolonged vibrations.
    \item Simulate real-world seismic or heavy traffic scenarios.
\end{itemize}

\subparagraph{How We Test It}
\begin{itemize}
    \item Use a vibration table to subject the lock to vibrations ranging 5–50 Hz at 0.5 g for 10 minutes.
    \item Observe physical condition and test lock/unlock function during and after vibration.
\end{itemize}

\subparagraph{Expectations of Test}
\begin{center}
\begin{tabular}{|c|p{10cm}|}
  \hline
  \textbf{Result} & \textbf{Conditions} \\
  \hline
  \textbf{PASS} &
    \begin{minipage}[t]{\linewidth}
    \begin{itemize}
      \item All mechanical parts remain in place and fully secure.
      \item Lock unlocks smoothly after vibration testing.
      \item No noise, rattling, or misalignment present.\\
    \end{itemize}
    \end{minipage} \\
  \hline
  \textbf{FAIL} & Any mechanical or electronic failure, loose components, or abnormal noises post-test. \\
  \hline
\end{tabular}
\end{center}
\end{samepage}

\newpage
\begin{samepage}
\subsection*{64. Extreme Cold Operation Test}

\subparagraph{Test Goals and Purpose}
\begin{itemize}
    \item Confirm reliable operation in extreme cold conditions as low as -40 °C.
    \item Assess material durability and battery performance under severe cold.
\end{itemize}

\subparagraph{How We Test It}
\begin{itemize}
    \item Place the lock in a temperature-controlled chamber set to -40 °C for 2 hours.
    \item Perform 10 full lock/unlock cycles while still in the chamber.
\end{itemize}

\subparagraph{Expectations of Test}
\begin{center}
\begin{tabular}{|c|p{10cm}|}
  \hline
  \textbf{Result} & \textbf{Conditions} \\
  \hline
  \textbf{PASS} &
    \begin{minipage}[t]{\linewidth}
    \begin{itemize}
      \item Lock mechanism operates without freezing or jamming.
      \item No visible cracking or material damage.
      \item Battery voltage remains stable, with normal MCU response.\\
    \end{itemize}
    \end{minipage} \\
  \hline
  \textbf{FAIL} & Any freezing, material cracking, or failure to operate as expected in the cold environment. \\
  \hline
\end{tabular}
\end{center}
\end{samepage}

\newpage
\begin{samepage}
\subsection*{65. Extreme Heat Operation Test}

\subparagraph{Test Goals and Purpose}
\begin{itemize}
    \item Assess the lock’s resilience when exposed to temperatures as high as 85 °C.
    \item Confirm no thermal-induced failures or performance degradation.
\end{itemize}

\subparagraph{How We Test It}
\begin{itemize}
    \item Place the lock in a heat chamber at 85 °C.
    \item Perform a lock/unlock cycle every 5 minutes for 2 hours to simulate repeated use.
\end{itemize}

\subparagraph{Expectations of Test}
\begin{center}
\begin{tabular}{|c|p{10cm}|}
  \hline
  \textbf{Result} & \textbf{Conditions} \\
  \hline
  \textbf{PASS} &
    \begin{minipage}[t]{\linewidth}
    \begin{itemize}
      \item No thermal shutdown or performance drop during high-temperature testing.
      \item Electronics, storage, and mechanical parts remain fully functional.
      \item No melting, warping, or deformation observed.\\
    \end{itemize}
    \end{minipage} \\
  \hline
  \textbf{FAIL} & Any performance issues, lock/unlock failures, or material deformation at high temperatures. \\
  \hline
\end{tabular}
\end{center}
\end{samepage}

\newpage
\begin{samepage}
\subsection*{66. Cosmic Ray Bit-Flip Test}

\subparagraph{Test Goals and Purpose}
\begin{itemize}
    \item Evaluate the system’s resilience to random memory errors caused by cosmic ray-induced bit flips.
    \item Ensure no critical system failures occur due to single-event upsets.
\end{itemize}

\subparagraph{How We Test It}
\begin{itemize}
    \item Use a software simulator to inject random bit flips into RAM and flash every 10 seconds.
    \item Observe the system for stability and recovery behavior over a 24-hour period.
\end{itemize}

\subparagraph{Expectations of Test}
\begin{center}
\begin{tabular}{|c|p{10cm}|}
  \hline
  \textbf{Result} & \textbf{Conditions} \\
  \hline
  \textbf{PASS} &
    \begin{minipage}[t]{\linewidth}
    \begin{itemize}
      \item System detects, corrects, or isolates bit errors without crashing.
      \item No data corruption in critical areas.
      \item Normal operation resumes automatically after error correction.\\
    \end{itemize}
    \end{minipage} \\
  \hline
  \textbf{FAIL} & Persistent system crashes, lockouts, or permanent data corruption. \\
  \hline
\end{tabular}
\end{center}
\end{samepage}


% 67


\newpage
\begin{samepage}
\subsection*{67. Wiring Pinch or Stress During Installation Test}

\subparagraph{Test Goals and Purpose}
\begin{itemize}
    \item Check if wires or cables get pinched or bent during installation on different doors.
    \item Make sure wiring stress doesn’t cause problems later.
    \item See if cable protection or strain relief features are needed.
\end{itemize}

\subparagraph{How We Test It}
\begin{itemize}
    \item Install the lock on several types of doors (wood, metal, composite).
    \item Watch how wires are positioned and if they get pinched or caught.
    \item Test lock function after installation to see if any wires are damaged or loose.
\end{itemize}

\subparagraph{Expectations of Test}
\begin{center}
\begin{tabular}{|c|p{10cm}|}
  \hline
  \textbf{Result} & \textbf{Conditions} \\
  \hline
  \textbf{PASS} &
    \begin{minipage}[t]{\linewidth}
    \begin{itemize}
      \item Wires are not pinched, cut, or stressed during installation.
      \item Lock works normally after install, with no random resets or signal errors.
      \item Cables are neat and protected from sharp edges.
    \end{itemize}
    \end{minipage} \\
  \hline
  \textbf{FAIL} & Wires are pinched or damaged during install, causing random errors or no unlock function. \\
  \hline
\end{tabular}
\end{center}
\end{samepage}



% 68


\newpage
\begin{samepage}
\subsection*{68. Internal Temperature Rise Under Heavy Use Test}

\subparagraph{Test Goals and Purpose}
\begin{itemize}
    \item Confirm that repeated usage of the lock does not cause overheating.
    \item Check that sensors and electronics keep working properly even when it’s warm inside the lock.
    \item See if adding software-based limits (like usage pauses) could be useful to avoid issues.
\end{itemize}

\subparagraph{How We Test It}
\begin{itemize}
    \item In a warm room, repeatedly unlock and lock the system 20 times in a row.
    \item Use a temperature sensor or thermal camera to watch internal temperatures.
    \item Look for delayed responses, sensor errors, or other abnormal behavior.
\end{itemize}

\subparagraph{Expectations of Test}
\begin{center}
\begin{tabular}{|c|p{10cm}|}
  \hline
  \textbf{Result} & \textbf{Conditions} \\
  \hline
  \textbf{PASS} &
    \begin{minipage}[t]{\linewidth}
    \begin{itemize}
      \item Internal temperature stays within safe range (e.g., below 60 degrees Celsius).
      \item Lock still unlocks and sensors report correctly.
      \item No errors or auto-shutdown triggered.
    \end{itemize}
    \end{minipage} \\
  \hline
  \textbf{FAIL} & Internal temperature gets too high and lock malfunctions, sensors report wrong data, or unlocks slow down. \\
  \hline
\end{tabular}
\end{center}
\end{samepage}


% 69


\newpage
\begin{samepage}
\subsection*{69. Battery Voltage Sag Under Load Test}

\subparagraph{Test Goals and Purpose}
\begin{itemize}
    \item Check how the lock’s electronics handle a temporary battery voltage drop during heavy use.
    \item Make sure there are no false “low battery” warnings or unexpected resets when the solenoid or motor is active.
    \item See if a voltage filter or hardware change could help improve reliability.
\end{itemize}

\subparagraph{How We Test It}
\begin{itemize}
    \item Fully charge the battery, then simulate repeated unlock commands to put load on the solenoid.
    \item Measure voltage at the battery terminals while activating the lock.
    \item Watch for system behavior: false warnings, resets, or unlock failures.
\end{itemize}

\subparagraph{Expectations of Test}
\begin{center}
\begin{tabular}{|c|p{10cm}|}
  \hline
  \textbf{Result} & \textbf{Conditions} \\
  \hline
  \textbf{PASS} &
    \begin{minipage}[t]{\linewidth}
    \begin{itemize}
      \item System ignores short-lived voltage drops and keeps working normally.
      \item No low-battery warning triggers unless battery is truly low.
      \item No resets, crashes, or unlock failures happen during high current draw.\\
    \end{itemize}
    \end{minipage} \\
  \hline
  \textbf{FAIL} & System triggers false low-battery alerts, resets, or fails to unlock when battery voltage dips briefly. \\
  \hline
\end{tabular}
\end{center}
\end{samepage}





