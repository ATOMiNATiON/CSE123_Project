\newpage
\begin{samepage}
\subsection*{43. Water Resistance Test}

\subparagraph{Test Goals and Purpose}
\begin{itemize}
    \item Verify that the lock's outer case and electronics remain operational when exposed to rain or moisture.
    \item Ensure that no short circuits or corrosion occur during or after water exposure.
    \item Confirm that the lock can still perform basic functions like unlocking and status reporting.
\end{itemize}

\subparagraph{How We Test It}
\begin{itemize}
    \item Simulate light rain by spraying water evenly over the lock case for 5 minutes.
    \item Observe lock functionality during and immediately after spraying:
    \begin{itemize}
        \item Check that the lock can still unlock using a valid PIN.
        \item Ensure no error messages or power resets occur.
    \end{itemize}
    \item After drying, inspect the internal electronics for any signs of water damage or corrosion.
\end{itemize}

\subparagraph{Expectations of Test}
\begin{center}
\begin{tabular}{|c|p{10cm}|}
  \hline
  \textbf{Result} & \textbf{Conditions} \\
  \hline
  \textbf{PASS} &
    \begin{minipage}[t]{\linewidth}
    \begin{itemize}
      \item Lock remains fully functional during and after water exposure.
      \item No electrical shorts, corrosion, or reset events occur.
      \item All mechanical and electronic components remain dry internally.\\
    \end{itemize}
    \end{minipage} \\
  \hline
  \textbf{FAIL} & Any sign of malfunction during rain simulation (e.g., lock won’t open, short circuits, or visible internal corrosion). \\
  \hline
\end{tabular}
\end{center}
\end{samepage}


% 44

\newpage
\begin{samepage}
\subsection*{44. Shock and Vibration Test}

\subparagraph{Test Goals and Purpose}
\begin{itemize}
    \item Ensure the lock case and internal electronics can withstand physical shocks and vibrations from door slams or impacts.
    \item Verify no functional or cosmetic damage occurs after simulated impacts.
    \item Confirm the lock can still unlock and report status accurately.
\end{itemize}

\subparagraph{How We Test It}
\begin{itemize}
    \item Apply mechanical shocks by gently striking the door or lock case with a rubber mallet.
    \item Simulate door vibration by rapidly opening and closing the door multiple times.
    \item Observe lock operation before, during, and after the tests:
    \begin{itemize}
        \item Enter a valid PIN and check if the lock unlocks.
        \item Look for physical cracks or alignment issues.
    \end{itemize}
\end{itemize}

\subparagraph{Expectations of Test}
\begin{center}
\begin{tabular}{|c|p{10cm}|}
  \hline
  \textbf{Result} & \textbf{Conditions} \\
  \hline
  \textbf{PASS} &
    \begin{minipage}[t]{\linewidth}
    \begin{itemize}
      \item No damage to lock case or internal parts.
      \item Lock unlocks normally with correct PIN after shocks and vibrations.
      \item No rattling or misalignment in case parts.\\
    \end{itemize}
    \end{minipage} \\
  \hline
  \textbf{FAIL} & Any malfunction or visible damage (e.g., broken parts, misalignment, or failed unlock test). \\
  \hline
\end{tabular}
\end{center}
\end{samepage}


% 45

\newpage
\begin{samepage}
\subsection*{45. Tamper Detection Test}

\subparagraph{Test Goals and Purpose}
\begin{itemize}
    \item Confirm that the lock’s tamper detection mechanism functions properly.
    \item Verify that tamper attempts (like partially opening the case) trigger the correct alerts.
    \item Ensure no false triggers during normal operation.
\end{itemize}

\subparagraph{How We Test It}
\begin{itemize}
    \item Slightly loosen or pry the case to simulate forced access.
    \item Observe lock behavior during and after this attempt:
    \begin{itemize}
        \item Check if LED indicator blinks or an alarm sounds.
        \item Verify that a “tamper event” is logged in Firestore if online.
    \end{itemize}
\end{itemize}

\subparagraph{Expectations of Test}
\begin{center}
\begin{tabular}{|c|p{10cm}|}
  \hline
  \textbf{Result} & \textbf{Conditions} \\
  \hline
  \textbf{PASS} &
    \begin{minipage}[t]{\linewidth}
    \begin{itemize}
      \item Tamper detection is triggered immediately when the case is pried.
      \item Alarm or LED indicator responds as intended.
      \item Event logged in Firestore if Wi-Fi is available.\\
    \end{itemize}
    \end{minipage} \\
  \hline
  \textbf{FAIL} & No alarm, no LED blink, or no Firestore log during tamper simulation. \\
  \hline
\end{tabular}
\end{center}
\end{samepage}

% 46


\newpage
\begin{samepage}
\subsection*{46. EMI (Electromagnetic Interference) Test}

\subparagraph{Test Goals and Purpose}
\begin{itemize}
    \item Verify that the lock continues to operate normally when exposed to common sources of RF noise.
    \item Ensure no false unlocking or failures occur due to nearby electromagnetic fields.
    \item Test for overall robustness in environments like commercial buildings with many RF devices.
\end{itemize}

\subparagraph{How We Test It}
\begin{itemize}
    \item Place RF devices (like a Wi-Fi router, handheld radio, or microwave) near the lock while it’s running.
    \item Enter a valid PIN and observe lock behavior for delays or malfunctions.
    \item Test communication with Firestore (if Wi-Fi is available) to ensure no RF interference.
\end{itemize}

\subparagraph{Expectations of Test}
\begin{center}
\begin{tabular}{|c|p{10cm}|}
  \hline
  \textbf{Result} & \textbf{Conditions} \\
  \hline
  \textbf{PASS} &
    \begin{minipage}[t]{\linewidth}
    \begin{itemize}
      \item No changes in lock response time or behavior.
      \item Lock still unlocks with valid PIN.
      \item Communication with Firestore (if available) remains stable.\\
    \end{itemize}
    \end{minipage} \\
  \hline
  \textbf{FAIL} & Lock fails to respond, false unlocks, or data issues during RF exposure. \\
  \hline
\end{tabular}
\end{center}
\end{samepage}


