\newpage
\begin{samepage}
\subsection*{43. Water Resistance Test}

\subparagraph{Test Goals and Purpose}
\begin{itemize}
    \item Verify that the lock's outer case and electronics remain operational when exposed to rain or moisture.
    \item Ensure that no short circuits or corrosion occur during or after water exposure.
    \item Confirm that the lock can still perform basic functions like unlocking and status reporting.
\end{itemize}

\subparagraph{How We Test It}
\begin{itemize}
    \item Simulate light rain by spraying water evenly over the lock case for 5 minutes.
    \item Observe lock functionality during and immediately after spraying:
    \begin{itemize}
        \item Check that the lock can still unlock using a valid PIN.
        \item Ensure no error messages or power resets occur.
    \end{itemize}
    \item After drying, inspect the internal electronics for any signs of water damage or corrosion.
\end{itemize}

\subparagraph{Expectations of Test}
\begin{center}
\begin{tabular}{|c|p{10cm}|}
  \hline
  \textbf{Result} & \textbf{Conditions} \\
  \hline
  \textbf{PASS} &
    \begin{minipage}[t]{\linewidth}
    \begin{itemize}
      \item Lock remains fully functional during and after water exposure.
      \item No electrical shorts, corrosion, or reset events occur.
      \item All mechanical and electronic components remain dry internally.\\
    \end{itemize}
    \end{minipage} \\
  \hline
  \textbf{FAIL} & Any sign of malfunction during rain simulation (e.g., lock won’t open, short circuits, or visible internal corrosion). \\
  \hline
\end{tabular}
\end{center}
\end{samepage}


% 44

\newpage
\begin{samepage}
\subsection*{44. Shock and Vibration Test}

\subparagraph{Test Goals and Purpose}
\begin{itemize}
    \item Ensure the lock case and internal electronics can withstand physical shocks and vibrations from door slams or impacts.
    \item Verify no functional or cosmetic damage occurs after simulated impacts.
    \item Confirm the lock can still unlock and report status accurately.
\end{itemize}

\subparagraph{How We Test It}
\begin{itemize}
    \item Apply mechanical shocks by gently striking the door or lock case with a rubber mallet.
    \item Simulate door vibration by rapidly opening and closing the door multiple times.
    \item Observe lock operation before, during, and after the tests:
    \begin{itemize}
        \item Enter a valid PIN and check if the lock unlocks.
        \item Look for physical cracks or alignment issues.
    \end{itemize}
\end{itemize}

\subparagraph{Expectations of Test}
\begin{center}
\begin{tabular}{|c|p{10cm}|}
  \hline
  \textbf{Result} & \textbf{Conditions} \\
  \hline
  \textbf{PASS} &
    \begin{minipage}[t]{\linewidth}
    \begin{itemize}
      \item No damage to lock case or internal parts.
      \item Lock unlocks normally with correct PIN after shocks and vibrations.
      \item No rattling or misalignment in case parts.\\
    \end{itemize}
    \end{minipage} \\
  \hline
  \textbf{FAIL} & Any malfunction or visible damage (e.g., broken parts, misalignment, or failed unlock test). \\
  \hline
\end{tabular}
\end{center}
\end{samepage}


% 45

\newpage
\begin{samepage}
\subsection*{45. Tamper Detection Test}

\subparagraph{Test Goals and Purpose}
\begin{itemize}
    \item Confirm that the lock’s tamper detection mechanism functions properly.
    \item Verify that tamper attempts (like partially opening the case) trigger the correct alerts.
    \item Ensure no false triggers during normal operation.
\end{itemize}

\subparagraph{How We Test It}
\begin{itemize}
    \item Slightly loosen or pry the case to simulate forced access.
    \item Observe lock behavior during and after this attempt:
    \begin{itemize}
        \item Check if LED indicator blinks or an alarm sounds.
        \item Verify that a “tamper event” is logged in Postgres if online.
    \end{itemize}
\end{itemize}

\subparagraph{Expectations of Test}
\begin{center}
\begin{tabular}{|c|p{10cm}|}
  \hline
  \textbf{Result} & \textbf{Conditions} \\
  \hline
  \textbf{PASS} &
    \begin{minipage}[t]{\linewidth}
    \begin{itemize}
      \item Tamper detection is triggered immediately when the case is pried.
      \item Alarm or LED indicator responds as intended.
      \item Event logged in Postgres if Wi-Fi is available.\\
    \end{itemize}
    \end{minipage} \\
  \hline
  \textbf{FAIL} & No alarm, no LED blink, or no Postgres log during tamper simulation. \\
  \hline
\end{tabular}
\end{center}
\end{samepage}

% 46


\newpage
\begin{samepage}
\subsection*{46. EMI (Electromagnetic Interference) Test}

\subparagraph{Test Goals and Purpose}
\begin{itemize}
    \item Verify that the lock continues to operate normally when exposed to common sources of RF noise.
    \item Ensure no false unlocking or failures occur due to nearby electromagnetic fields.
    \item Test for overall robustness in environments like commercial buildings with many RF devices.
\end{itemize}

\subparagraph{How We Test It}
\begin{itemize}
    \item Place RF devices (like a Wi-Fi router, handheld radio, or microwave) near the lock while it’s running.
    \item Enter a valid PIN and observe lock behavior for delays or malfunctions.
    \item Test communication with Postgres (if Wi-Fi is available) to ensure no RF interference.
\end{itemize}

\subparagraph{Expectations of Test}
\begin{center}
\begin{tabular}{|c|p{10cm}|}
  \hline
  \textbf{Result} & \textbf{Conditions} \\
  \hline
  \textbf{PASS} &
    \begin{minipage}[t]{\linewidth}
    \begin{itemize}
      \item No changes in lock response time or behavior.
      \item Lock still unlocks with valid PIN.
      \item Communication with Postgres (if available) remains stable.\\
    \end{itemize}
    \end{minipage} \\
  \hline
  \textbf{FAIL} & Lock fails to respond, false unlocks, or data issues during RF exposure. \\
  \hline
\end{tabular}
\end{center}
\end{samepage}



% 47

\newpage
\begin{samepage}
\subsection*{47. Battery Swapping Time Test}

\subparagraph{Test Goals and Purpose}
\begin{itemize}
    \item Make sure that a depleted battery can be swapped out quickly and safely by the user.
    \item Confirm the lock system operates normally and immediately after a fresh battery.
    \item Check that no errors or abnormal behavior occur during or after the swap.
\end{itemize}

\subparagraph{How We Test It}
\begin{itemize}
    \item Simulate battery depletion until the lock no longer responds.
    \item Swap in a fully charged battery, measuring the time until lock operates again.
    \item Verify the lock can accept PIN entry and respond to app commands immediately.
    \item Check Postgres or local logs for “POWER\_RESTORED” event and normal lock state.
\end{itemize}

\subparagraph{Expectations of Test}
\begin{center}
\begin{tabular}{|c|p{10cm}|}
  \hline
  \textbf{Result} & \textbf{Conditions} \\
  \hline
  \textbf{PASS} &
    \begin{minipage}[t]{\linewidth}
    \begin{itemize}
      \item Fresh battery swap is completed within 10 seconds.
      \item Lock accepts valid PIN entry and responds to app commands immediately after swap.
      \item No errors, resets, or abnormal delays are observed.
      \item Postgres or system logs event with correct timestamp.\\
    \end{itemize}
    \end{minipage} \\
  \hline
  \textbf{FAIL} & Any of the PASS conditions are not met, or lock remains unresponsive after swap. \\
  \hline
\end{tabular}
\end{center}
\end{samepage}


% 48


\newpage
\begin{samepage}
\subsection*{48. Battery Cycle Life Test}

\subparagraph{Test Goals and Purpose}
\begin{itemize}
    \item Assess how many charge/discharge cycles the battery can endure before significant capacity loss occurs.
    \item Confirm the battery meets lifespan expectations for typical usage scenarios.
    \item Evaluate any system degradation from aging batteries.
\end{itemize}

\subparagraph{How We Test It}
\begin{itemize}
    \item Fully charge the battery and measure its capacity (e.g., using a battery analyzer or capacity measurement tool).
    \item Discharge the battery completely through fast discharge source.
    \item Repeat this charge-discharge cycle 300 times (or manufacturer’s recommended cycle count).
    \item After 300 cycles, measure the battery capacity again and compare to initial measurement.
    \item Observe lock performance during and after these cycles.
\end{itemize}

\subparagraph{Expectations of Test}
\begin{center}
\begin{tabular}{|c|p{10cm}|}
  \hline
  \textbf{Result} & \textbf{Conditions} \\
  \hline
  \textbf{PASS} &
    \begin{minipage}[t]{\linewidth}
    \begin{itemize}
      \item Battery retains at least 80\% of its original capacity after 300 cycles.
      \item No lock performance degradation or instability occurs during normal operation.
      \item Post-test capacity and logs indicate continued reliable operation.\\
    \end{itemize}
    \end{minipage} \\
  \hline
  \textbf{FAIL} & Capacity drops below 80\% or lock experiences malfunctions during or after cycling. \\
  \hline
\end{tabular}
\end{center}
\end{samepage}


% 49



\newpage
\subsection*{49. RTC Drift Test After Offline Period}
\subparagraph{Test Goals and Purpose}
\begin{itemize}
    \item Measure the time drift of the ESP32’s RTC after prolonged offline operation (e.g., one week without network time synchronization).
    \item Assess whether the RTC accuracy meets system requirements for timestamping events and logs.
    \item Determine if corrective measures (e.g., periodic NTP sync) are necessary.
\end{itemize}

\subparagraph{How We Test It}
\begin{itemize}
    \item Synchronize the ESP32’s RTC to accurate time (e.g., via NTP).
    \item Disconnect the device from the network and keep powered for one week.
    \item After one week, compare the ESP32 RTC time against a reference clock (e.g., atomic clock, smartphone).
    \item Record the time difference (drift) in minutes.
\end{itemize}

\subparagraph{Expectations of Test}
\begin{center}
    \begin{tabular}{|c|p{10cm}|}
      \hline
      \textbf{Result} & \textbf{Conditions} \\
      \hline
      \textbf{PASS} & 
        \begin{minipage}[t]{\linewidth}
        \begin{itemize}
          \item RTC drift is less than or equal to 1 minute after one week offline.
          \item System logs reflect consistent and accurate timestamps.
          \item No significant impact on event timing or logging accuracy.\\
        \end{itemize}
        \end{minipage} \\
      \hline
      \textbf{FAIL} & Any one of the PASS conditions is missing or incorrect. \\
      \hline
    \end{tabular}
\end{center}



% 50



\newpage
\begin{samepage}
\subsection*{50. Partial PIN Entry Timeout Test}

\subparagraph{Test Goals and Purpose}
\begin{itemize}
    \item Verify that if a user starts entering a PIN but stops, the system eventually clears the partial entry.
    \item Confirm that incomplete PIN entries do not accidentally trigger a false unlock or other errors.
\end{itemize}

\subparagraph{How We Test It}
\begin{itemize}
    \item Enter the first 2 digits of a valid 4-digit PIN and then wait for 30 seconds without completing it.
    \item Observe whether the system automatically clears the partial entry.
    \item Enter a valid PIN after the timeout period to confirm the system still works.
\end{itemize}

\subparagraph{Expectations of Test}
\begin{center}
\begin{tabular}{|c|p{10cm}|}
  \hline
  \textbf{Result} & \textbf{Conditions} \\
  \hline
  \textbf{PASS} &
    \begin{minipage}[t]{\linewidth}
    \begin{itemize}
      \item Partial PIN entry clears automatically after 30 seconds of no input.
      \item No unintended unlock or errors happen during partial entry.
      \item Entering a complete valid PIN after timeout unlocks the door as normal.\\
    \end{itemize}
    \end{minipage} \\
  \hline
  \textbf{FAIL} & Partial PIN doesn’t clear on its own, or a partial PIN causes an unintended unlock or error. \\
  \hline
\end{tabular}
\end{center}
\end{samepage}


% 51


\newpage
\begin{samepage}
\subsection*{51. Short Repeated Unlock Commands Test}

\subparagraph{Test Goals and Purpose}
\begin{itemize}
    \item Ensure that the system does not try to unlock repeatedly if multiple unlock commands come in very quickly.
    \item Confirm that only the first unlock command executes and no confusion or extra mechanical wear happens.
\end{itemize}

\subparagraph{How We Test It}
\begin{itemize}
    \item Use two phones or apps to send unlock commands within 2 seconds of each other.
    \item Observe the bolt movement and system response.
    \item Check that only one unlock movement occurs and both commands are logged correctly.
\end{itemize}

\subparagraph{Expectations of Test}
\begin{center}
\begin{tabular}{|c|p{10cm}|}
  \hline
  \textbf{Result} & \textbf{Conditions} \\
  \hline
  \textbf{PASS} &
    \begin{minipage}[t]{\linewidth}
    \begin{itemize}
      \item Only one unlock actuation occurs for the rapid commands.
      \item Both commands are logged in the database or logs.
      \item No duplicate unlocking motions, error logs, or misalignment of bolt position.\\
    \end{itemize}
    \end{minipage} \\
  \hline
  \textbf{FAIL} & Lock actuates multiple times or errors are seen during rapid unlock commands. \\
  \hline
\end{tabular}
\end{center}
\end{samepage}



% 52


\newpage
\begin{samepage}
\subsection*{52. Battery Contact Flicker Test}

\subparagraph{Test Goals and Purpose}
\begin{itemize}
    \item Check how the system handles momentary loss of power if the battery contacts flicker or become loose.
    \item Make sure the system recovers cleanly and no permanent errors happen.
\end{itemize}

\subparagraph{How We Test It}
\begin{itemize}
    \item Gently wiggle the battery to simulate a brief disconnect or flicker.
    \item Observe whether the lock resets, logs a power event, and recovers.
    \item Try to unlock using a valid PIN after the power flicker.
\end{itemize}

\subparagraph{Expectations of Test}
\begin{center}
\begin{tabular}{|c|p{10cm}|}
  \hline
  \textbf{Result} & \textbf{Conditions} \\
  \hline
  \textbf{PASS} &
    \begin{minipage}[t]{\linewidth}
    \begin{itemize}
      \item System safely resets if power flickers and fully recovers.
      \item No error logs or false unlocks happen because of the flicker.
      \item Valid PIN unlock works after power recovers.\\
    \end{itemize}
    \end{minipage} \\
  \hline
  \textbf{FAIL} & System crashes, lock stays stuck, or partial unlock occurs during battery flicker. \\
  \hline
\end{tabular}
\end{center}
\end{samepage}



% 53



\newpage
\begin{samepage}
\subsection*{53. Partial Keypad Failure Test}

\subparagraph{Test Goals and Purpose}
\begin{itemize}
    \item Check how the lock system responds if one of the keypad buttons is physically stuck or not working.
    \item Make sure that other keys still work and that the system does not unlock unexpectedly.
    \item Confirm proper error handling for partial keypad failure.
\end{itemize}

\subparagraph{How We Test It}
\begin{itemize}
    \item Simulate a stuck or non-functional key by taping it down or disconnecting that key's wire (if possible).
    \item Try to enter a valid PIN that requires the failed key.
    \item Then try a PIN that does not use the failed key.
    \item Observe if lock behavior is consistent and safe.
\end{itemize}

\subparagraph{Expectations of Test}
\begin{center}
\begin{tabular}{|c|p{10cm}|}
  \hline
  \textbf{Result} & \textbf{Conditions} \\
  \hline
  \textbf{PASS} &
    \begin{minipage}[t]{\linewidth}
    \begin{itemize}
      \item Lock does not unlock with partial PINs or if the broken key is needed.
      \item Other keys work normally and unaffected PINs unlock as expected.
      \item System does not produce random errors or unlock with bad PINs.\\
    \end{itemize}
    \end{minipage} \\
  \hline
  \textbf{FAIL} & Lock unlocks even when an incorrect PIN is entered due to keypad error, or system becomes unresponsive. \\
  \hline
\end{tabular}
\end{center}
\end{samepage}


% 54


\newpage
\begin{samepage}
\subsection*{54. Nearby Electromagnetic Fields Test}

\subparagraph{Test Goals and Purpose}
\begin{itemize}
    \item Make sure that nearby electromagnetic interference does not cause false unlocks or errors.
    \item Verify the lock keeps working normally when in areas with high RF noise (like near a microwave or big power transformer).
\end{itemize}

\subparagraph{How We Test It}
\begin{itemize}
    \item Place the lock within 0.5 m of an active microwave oven or other RF source.
    \item Send commands and try keypad unlocks while device is exposed.
    \item Observe if the lock still works and no false signals are seen.
\end{itemize}

\subparagraph{Expectations of Test}
\begin{center}
\begin{tabular}{|c|p{10cm}|}
  \hline
  \textbf{Result} & \textbf{Conditions} \\
  \hline
  \textbf{PASS} &
    \begin{minipage}[t]{\linewidth}
    \begin{itemize}
      \item No false unlocks or random keypad responses.
      \item Unlocks and commands work normally without delay or errors.
      \item System logs show no interference warnings or error events.\\
    \end{itemize}
    \end{minipage} \\
  \hline
  \textbf{FAIL} & Lock unlocks by itself, shows errors, or stops responding during RF exposure. \\
  \hline
\end{tabular}
\end{center}
\end{samepage}


% 55


\newpage
\begin{samepage}
\subsection*{55. Dust or Debris in Keypad Test}

\subparagraph{Test Goals and Purpose}
\begin{itemize}
    \item Make sure that small amounts of dust, dirt, or moisture on the keypad do not stop it from working.
    \item Confirm no random unlocks or false signals happen because of dirt.
\end{itemize}

\subparagraph{How We Test It}
\begin{itemize}
    \item Sprinkle a small amount of dust or dirt on the keypad.
    \item Enter a valid PIN and see if the lock responds normally.
    \item Clean off the keypad and test again to make sure it still works as before.
\end{itemize}

\subparagraph{Expectations of Test}
\begin{center}
\begin{tabular}{|c|p{10cm}|}
  \hline
  \textbf{Result} & \textbf{Conditions} \\
  \hline
  \textbf{PASS} &
    \begin{minipage}[t]{\linewidth}
    \begin{itemize}
      \item Valid PINs still unlock the lock even with a little dirt.
      \item No false unlocks or random entries are seen.
      \item Cleaning the keypad fully restores normal performance.\\
    \end{itemize}
    \end{minipage} \\
  \hline
  \textbf{FAIL} & Keypad stops responding, triggers false unlocks, or logs errors because of dirt. \\
  \hline
\end{tabular}
\end{center}
\end{samepage}



% 56


\newpage
\begin{samepage}
\subsection*{56. Keypad Button Wear Test}

\subparagraph{Test Goals and Purpose}
\begin{itemize}
    \item Verify that repeated use of the same button doesn't cause it to fail or degrade performance.
    \item Make sure the keypad can handle frequent presses in real-world usage.
\end{itemize}

\subparagraph{How We Test It}
\begin{itemize}
    \item Use a mechanical tester (or finger) to press the same button at least 5000 times.
    \item After testing, enter a valid PIN that uses that button and see if it still works normally.
    \item Also test that other buttons on the keypad still function without issue.
\end{itemize}

\subparagraph{Expectations of Test}
\begin{center}
\begin{tabular}{|c|p{10cm}|}
  \hline
  \textbf{Result} & \textbf{Conditions} \\
  \hline
  \textbf{PASS} &
    \begin{minipage}[t]{\linewidth}
    \begin{itemize}
      \item Key still works after 5000 presses.
      \item Valid PINs that use the worn key still unlock correctly.
      \item No sticky keys or false inputs occur after the test.\\
    \end{itemize}
    \end{minipage} \\
  \hline
  \textbf{FAIL} & Key fails to register presses, or valid PINs can’t unlock after wear test. \\
  \hline
\end{tabular}
\end{center}
\end{samepage}


% 57


\newpage
\begin{samepage}
\subsection*{57. Bolt Retraction Minimum Voltage Test}

\subparagraph{Test Goals and Purpose}
\begin{itemize}
    \item Check the lowest voltage where the solenoid bolt still retracts reliably.
    \item Confirm that unlocking works even when the battery is getting low.
\end{itemize}

\subparagraph{How We Test It}
\begin{itemize}
    \item Use a power supply to set the voltage of the system.
    \item Slowly reduce voltage from nominal (e.g., 12 V) down in steps (e.g., 0.5 V).
    \item At each step, send an unlock command and see if bolt retracts fully.
\end{itemize}

\subparagraph{Expectations of Test}
\begin{center}
\begin{tabular}{|c|p{10cm}|}
  \hline
  \textbf{Result} & \textbf{Conditions} \\
  \hline
  \textbf{PASS} &
    \begin{minipage}[t]{\linewidth}
    \begin{itemize}
      \item Bolt retracts fully down to the minimum voltage (as per product spec).
      \item Unlocks smoothly, no partial or stuck motion.
      \item System logs or status remain normal (no errors, no misalignment).\\
    \end{itemize}
    \end{minipage} \\
  \hline
  \textbf{FAIL} & Bolt fails to fully retract or system errors occur before minimum voltage limit. \\
  \hline
\end{tabular}
\end{center}
\end{samepage}



% 58



\newpage
\begin{samepage}
\subsection*{58. ESP to Solenoid Relay Stress Test}

\subparagraph{Test Goals and Purpose}
\begin{itemize}
    \item Verify that repeated relay switching by the ESP32 doesn’t cause damage or misfires.
    \item Ensure the relay coil and driver can handle frequent switching.
\end{itemize}

\subparagraph{How We Test It}
\begin{itemize}
    \item Command the ESP32 to toggle the solenoid relay 500 times rapidly (like 1 Hz or faster).
    \item Watch for stuck relay, missed commands, or system resets.
    \item Confirm the lock still functions normally after the stress test.
\end{itemize}

\subparagraph{Expectations of Test}
\begin{center}
\begin{tabular}{|c|p{10cm}|}
  \hline
  \textbf{Result} & \textbf{Conditions} \\
  \hline
  \textbf{PASS} &
    \begin{minipage}[t]{\linewidth}
    \begin{itemize}
      \item Relay toggles correctly every time.
      \item No sticking or missed commands.
      \item Lock responds normally to unlock/lock commands after the test.
      \item No heating or damage to relay or ESP32 driver pins.\\
    \end{itemize}
    \end{minipage} \\
  \hline
  \textbf{FAIL} & Relay fails to toggle, sticks, or lock fails after test due to hardware damage. \\
  \hline
\end{tabular}
\end{center}
\end{samepage}


% 59




\newpage
\subsection*{59. I²C Temperature Sensor Fault Tolerance Test}
\subparagraph{Test Goals and Purpose}
\begin{itemize}
    \item Verify the system handles brief communication errors or invalid data from the temperature sensor gracefully.
    \item Ensure no crashes, system hangs, or misleading readings are shown to the user.
    \item Confirm that erroneous data is filtered or flagged appropriately.
\end{itemize}

\subparagraph{How We Test It}
\begin{itemize}
    \item Emulate an I²C fault using a glitch generator, cable disturbance, or by temporarily disconnecting the sensor.
    \item Allow the sensor to resume normal operation after 1–2 cycles.
    \item Observe system logs, display/app data, and Postgres fields related to temperature.
    \item Confirm whether system handles retry logic or flags invalid reads.
\end{itemize}

\subparagraph{Expectations of Test}
\begin{center}
    \begin{tabular}{|c|p{10cm}|}
      \hline
      \textbf{Result} & \textbf{Conditions} \\
      \hline
      \textbf{PASS} & 
        \begin{minipage}[t]{\linewidth}
        \begin{itemize}
          \item System does not crash, hang, or misbehave when invalid data is received.
          \item Invalid temperature readings are ignored or marked as “error” in logs.
          \item Postgres/app either retains last valid value or displays "unavailable".
          \item Normal readings resume once sensor stabilizes.\\
        \end{itemize}
        \end{minipage} \\
      \hline
      \textbf{FAIL} & Any one of the PASS conditions is missing or incorrect. \\
      \hline
    \end{tabular}
\end{center}



% 60


\newpage
\begin{samepage}
\subsection*{60. OTA Update Failure and Rollback Test}
\subparagraph{Test Goals and Purpose}
\begin{itemize}
    \item Verify that the smart lock safely handles OTA update interruptions during mid-download.
    \item Confirm the system automatically rolls back to the prior stable firmware version.
    \item Ensure no corruption or bricking occurs, maintaining device operability.
\end{itemize}

\subparagraph{How We Test It}
\begin{itemize}
    \item Initiate an OTA firmware update.
    \item Simulate failure by interrupting the download mid-transfer (e.g., disconnect network or power).
    \item Power cycle the device and observe boot behavior.
    \item Verify the device boots the previous firmware version without errors.
    \item Test normal lock/unlock functionality after rollback.
\end{itemize}

\subparagraph{Expectations of Test}
\begin{center}
    \begin{tabular}{|c|p{10cm}|}
      \hline
      \textbf{Result} & \textbf{Conditions} \\
      \hline
      \textbf{PASS} & 
        \begin{minipage}[t]{\linewidth}
        \begin{itemize}
          \item Device detects incomplete or corrupted OTA update.
          \item Automatically rolls back to the previous firmware version.
          \item Device boots successfully without entering a bricked state.
          \item Lock functions correctly with prior firmware after rollback.
          \item System logs the failed update and rollback event.\\
        \end{itemize}
        \end{minipage} \\
      \hline
      \textbf{FAIL} & Any one of the PASS conditions is missing or incorrect. \\
      \hline
    \end{tabular}
\end{center}
\end{samepage}


% 61



\newpage
\begin{samepage}

\subsection*{61. SPI Flash Corruption and Boot Configuration Error Handling Test}
\subparagraph{Test Goals and Purpose}
\begin{itemize}
    \item Verify the system can detect and handle corrupted SPI flash data.
    \item Ensure the device fails gracefully and does not enter an unstable or undefined state.
    \item Confirm that the firmware provides recovery options, such as default fallback config or safe mode.
    \item Prevent bricking or incorrect behavior due to unreadable configuration data.
\end{itemize}

\subparagraph{How We Test It}
\begin{itemize}
    \item Intentionally corrupt the configuration section of the SPI flash (e.g., overwrite with invalid or random data).
    \item Reboot the smart lock and observe its behavior during initialization.
    \item Check system logs and debug output for detection and reporting of the corruption.
    \item Evaluate whether the system uses fallback defaults or enters a known safe state.
\end{itemize}

\subparagraph{Expectations of Test}
\begin{center}
    \begin{tabular}{|c|p{10cm}|}
      \hline
      \textbf{Result} & \textbf{Conditions} \\
      \hline
      \textbf{PASS} & 
        \begin{minipage}[t]{\linewidth}
        \begin{itemize}
          \item Firmware detects SPI flash corruption during boot.
          \item Lock enters safe mode or loads default configuration.
          \item Device does not crash, freeze, or act unpredictably.
          \item Logs include clear error message for corrupted config.
          \item Optional: App or user is notified to reconfigure settings.\\
        \end{itemize}
        \end{minipage} \\
      \hline
      \textbf{FAIL} & Any one of the PASS conditions is missing or incorrect. \\
      \hline
    \end{tabular}
\end{center}
\end{samepage}


% 62

\newpage
\begin{samepage}
\subsection*{62. Direct Cloud Modification Rejection Test}

\subparagraph{Test Goals and Purpose}
\begin{itemize}
    \item Ensure that changing lock state directly in the cloud database does not unlock or lock the door.
    \item Verify that only the app or authorized hardware commands can control the lock.
\end{itemize}

\subparagraph{How We Test It}
\begin{itemize}
    \item Manually change the \texttt{LOCK\_STATE} field in the cloud database to 1 (unlock) or 0 (lock).
    \item Observe if the lock physically responds to this direct change.
    \item Send an unlock/lock command from the app afterward to verify normal behavior.
\end{itemize}

\subparagraph{Expectations of Test}
\begin{center}
\begin{tabular}{|c|p{10cm}|}
  \hline
  \textbf{Result} & \textbf{Conditions} \\
  \hline
  \textbf{PASS} &
    \begin{minipage}[t]{\linewidth}
    \begin{itemize}
      \item Lock does not physically move when \texttt{LOCK\_STATE} is changed directly in the cloud.
      \item Only commands from app or keypad trigger real mechanical movement.
      \item Logs show no unintended actions during direct cloud change.\\
    \end{itemize}
    \end{minipage} \\
  \hline
  \textbf{FAIL} & Lock physically unlocks/locks because of direct cloud edits without app/hardware command. \\
  \hline
\end{tabular}
\end{center}
\end{samepage}



% 63


\newpage
\begin{samepage}
\subsection*{63. Delayed Unlock Feedback Test}

\subparagraph{Test Goals and Purpose}
\begin{itemize}
    \item Confirm that if there’s a slight delay in unlocking (like due to slow WiFi or cloud latency), the app still shows a “pending” or waiting indicator.
    \item Prevent confusion for the user who might think the unlock failed when it’s just slow.
\end{itemize}

\subparagraph{How We Test It}
\begin{itemize}
    \item Intentionally slow down the WiFi network or cloud server response.
    \item Send an unlock command from the app.
    \item Observe the app display and see if it shows “pending” or similar feedback.
\end{itemize}

\subparagraph{Expectations of Test}
\begin{center}
\begin{tabular}{|c|p{10cm}|}
  \hline
  \textbf{Result} & \textbf{Conditions} \\
  \hline
  \textbf{PASS} &
    \begin{minipage}[t]{\linewidth}
    \begin{itemize}
      \item App clearly shows that unlock is pending or waiting.
      \item Once delay resolves, app shows success or failure based on actual result.
      \item No confusion or misleading message in the app UI.\\
    \end{itemize}
    \end{minipage} \\
  \hline
  \textbf{FAIL} & App shows wrong status (like immediate success or nothing at all) during network delay. \\
  \hline
\end{tabular}
\end{center}
\end{samepage}



% 64


\newpage
\begin{samepage}
\subsection*{64. WiFi Channel Interference Test}

\subparagraph{Test Goals and Purpose}
\begin{itemize}
    \item Verify that nearby WiFi networks on the same or overlapping channels do not prevent the lock from responding to commands.
    \item Ensure lock can handle a typical home environment with many devices.
\end{itemize}

\subparagraph{How We Test It}
\begin{itemize}
    \item Set up nearby WiFi access points on the same channel as the lock’s WiFi.
    \item Try sending lock/unlock commands via app and observe behavior.
    \item Repeat with different channels to compare.
\end{itemize}

\subparagraph{Expectations of Test}
\begin{center}
\begin{tabular}{|c|p{10cm}|}
  \hline
  \textbf{Result} & \textbf{Conditions} \\
  \hline
  \textbf{PASS} &
    \begin{minipage}[t]{\linewidth}
    \begin{itemize}
      \item Lock responds to unlock/lock commands normally despite interference.
      \item No lost commands or random behavior seen.
      \item Logs show no errors or retries triggered by interference.\\
    \end{itemize}
    \end{minipage} \\
  \hline
  \textbf{FAIL} & Lock fails to respond, delays are excessive, or system logs show errors due to WiFi overlap. \\
  \hline
\end{tabular}
\end{center}
\end{samepage}


% 65



\newpage
\subsection*{65. Power Surge Resilience Test}
\subparagraph{Test Goals and Purpose}
\begin{itemize}
    \item Ensure the ESP32 safely handles sudden voltage spikes or drops (brown-out conditions).
    \item Verify that the system resets safely without hardware damage.
    \item Confirm that the device automatically resumes correct operation after reset.
\end{itemize}
\subparagraph{How We Test It}
\begin{itemize}
    \item Connect the device to a programmable power supply.
    \item Apply a rapid voltage increase (e.g., spike to 4.5V) or decrease (e.g., drop below 3.0V) to simulate surge or brown-out.
    \item Observe if the ESP32 performs a brown-out reset.
    \item Monitor recovery behavior and system logs.
\end{itemize}
\subparagraph{Expectations of Test}
\begin{center}
    \begin{tabular}{|c|p{10cm}|}
      \hline
      \textbf{Result} & \textbf{Conditions} \\
      \hline
      \textbf{PASS} &
        \begin{minipage}[t]{\linewidth}
        \begin{itemize}
          \item ESP32 performs a brown-out reset when voltage is out of range.
          \item No permanent hardware damage occurs.
          \item Device reconnects to Wi-Fi and Postgres automatically.
          \item System resumes normal operation within 10 seconds of recovery. \\
        \end{itemize}
        \end{minipage} \\
      \hline
      \textbf{FAIL} & Any one of the PASS conditions is missing or incorrect. \\
      \hline
    \end{tabular}
    \end{center}
