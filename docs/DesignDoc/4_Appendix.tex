\newpage
\subsection{Appendix 4 - Review}
    \subsubsection*{Nathaniel Laurente}
    Throughout the quarter, designing our smart lock proved very difficult when delving into the details of each component. I believe starting on the prototype early went very well for us since we were able to do our demonstration with all the features we wanted to prove through it. Researching how to design our manufactured product was difficult. We didn’t realize how hard it would be to find and design parts for an actual industry grade product, nor really delve into many different edge cases to see if these parts fit our needs to pass certain tests. We managed to find all the right parts, database, tools, manufacturing processes, and software needed to produce what we thought could be a potentially marketable product. As a team, it was an interesting experience to set up tasks on our GitHub, and coordinate times for meeting up or completing certain aspects of the final design and prototype.

    \subsubsection*{Adam Wu}
    Over the course of this project, I learned a lot-especially about teamwork and how to effectively use GitHub for collaboration. I enjoyed the process of problem-solving and working through challenges as a team. The struggle itself was a valuable part of the experience. One of the things that went well was how we supported each other and made steady progress, especially when we were able to meet in person. However one of the biggest challenges we faced was coordinating our schedules to find time to work together physically. We found that our most productive and collaborative moments happened when we were all in the same space, so scheduling conflicts often slowed down our momentum. Despite that, I’m happy with what we accomplished being the smallest team and how much I’ve grown from the experience.

    \subsubsection*{Jackson Kennedy}
    For the duration of this class, I found myself running into problems along the way both programmatically as well as problems with finding times to meet as a group with conflicting schedules and life events. Regardless, we were able to power through these problems and fix code, as well as find common times that when we all were together, proved monumental in progressing the overall project. If we were to do this class over, I think that finding and enforcing a more common schedule where we can all meet in person would be ideal because some of the best work happened when we were all present. One un-avoidable thing was having a changing group size and project path but I believe we handled the change well and were able to deliver on the product scope which made the class very rewarding.

    \subsubsection*{Neena Nguyen}
    One of the most rewarding parts of this project was getting to work closely with my team. We regularly met in person and on Discord to troubleshoot issues, share updates, and figure things out together. Even when things got difficult, especially when deadlines were tight, I enjoyed problem-solving as a unit, not just individually. I also really enjoyed designing the posters and presentation materials for our group. The most rewarding part of this experience was seeing how our ideas slowly turned into something real. We started with a mess of wires and components on a breadboard, and by the end, we had a functioning smart lock prototype with a physical enclosure. Even though there were major technical challenges for me and I was pushed to learn a lot, what stood out most was how much we grew as a team throughout the process. We each had our own roles, but we also made space for each other to give input and ask for help. I felt like I had ownership over specific parts of the project, but never felt like I was working in isolation. If I were to do this again, I’d ask for more guidance early on in the areas I was less familiar with. I’d also try to plan ahead more for the database structure. But overall, I’m proud of how we communicated, adapted, and built something meaningful together.