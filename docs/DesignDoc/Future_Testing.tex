\newpage
\begin{samepage}
    \section{Manufactured Product Testing}
    \subsection{Common Scenarios (Tests 1–10)}
    % ---------------------------- Test 1 --------------------------------
    \subsection*{1. Valid PIN Entry Test}
    \subparagraph{Test Goals and Purpose}
    \begin{itemize}
        \item Verify that a recognized 4-digit PIN unlocks the door without unnecessary delay.
        \item Check that the system logs this event as a successful access in Firestore.
    \end{itemize}
    \subparagraph{How We Test It}
    \begin{enumerate}
        \item Select a PIN known to be valid from the internal list.
        \item Enter the PIN once on the keypad.
        \item Observe the system's response, checking for physical bolt movement and UI changes on the mobile application.
        \item Look for updates to the (\texttt{LOCK\_STATE}) value in Firestore.
    \end{enumerate}
    
    \textbf{Expectations of Test}
    \begin{center}
    \begin{tabular}{|c|p{10cm}|}
      \hline
      \textbf{Result} & \textbf{Conditions} \\
      \hline
      PASS & Bolt retracts within \textless{}0.5\,s; \texttt{LOCK\_STATE}=1; Firestore log entry \emph{success}. \\
      \hline
      FAIL & Any deviation from the above timing, state, or logging behaviour. \\
      \hline
    \end{tabular}
    \end{center}
\end{samepage}


% ---------------------------- Test 2 --------------------------------
\newpage
\subsection*{2. Invalid PIN Lockout Test}
\subparagraph{Test Goals and Purpose}
\begin{itemize}
    \item Ensure the system prevents further PIN attempts after multiple consecutive failures.
    \item Lock should be shut down for a short expiration period.
    \item Confirm some form of lockout indication is presented, such as a light or warning beep.
    \item Owner's phoone should receive an alert about the failed attempts.
\end{itemize}
\subparagraph{How We Test It}
\begin{itemize}
    \item Enter three clearly incorrect PINs in under 30 seconds.
    \item Observe for a visible or audible signal.
    \item Confirm that \texttt{LOCKOUT\_COUNT} has risen and \texttt{LOCK\_STATE} is still 0 (locked).
\end{itemize}
\subparagraph{Expectations of Test}
\begin{center}
\begin{tabular}{|c|p{10cm}|}
  \hline
  \textbf{Result} & \textbf{Conditions} \\
  \hline
  \textbf{PASS} &
    \begin{minipage}[t]{\linewidth}
    \begin{itemize}
      \item Keypad flashes red and/or beeps immediately after the \textbf{third} wrong entry.
      \item No further PINs are accepted for the lockout period (default: 60 s).
      \item Mobile app log shows “3 invalid PIN attempts – lock is temporarily disabled” within 5 s.
      \item Firestore records \texttt{LOCKOUT\_COUNT = 3} and \texttt{LOCK\_STATE = 0}.\\
    \end{itemize}
    \end{minipage} \\ 
  \hline
  \textbf{FAIL} & Any one of the PASS conditions is missing or incorrect. \\ 
  \hline
\end{tabular}
\end{center}
\vspace{0.5em}

\noindent\textbf{Reset after the test:}  
Either wait for the lockout timer to expire or enter an admin PIN to clear the lockout so the next test starts from a clean state.\\


% ---------------------------- Test 3 --------------------------------
\newpage
\begin{samepage}
    \subsection*{3. Emergency PIN Test}
    \subparagraph{Test Goals and Purpose}
    \begin{itemize}
        \item Verify that a designated emergency PIN can override an active lockout state.
        \item Ensure that using the emergency PIN resets lockout counters and restores normal operation.
        \item Confirm the system logs the emergency unlock event.
    \end{itemize}
    \subparagraph{How We Test It}
    \begin{itemize}
        \item Trigger a lockout by intentionally inputting three incorrect PIN entries.
        \item Confirm after each incorrect PIN attempt (\texttt{LOCK\_STATE} = 1).
        \item Immediately enter the emergency PIN once the lockout state is active.
        \item Observe the lock mechanism (it should retract).
        \item Monitor Firestore to verify an EMERGENCY\_UNLOCK entry is created.
        \item Owner's phone should be notified and the mobile application should display that the emergency PIN was used.
        \item In Firestore, confirm that \texttt{LOCKOUT\_COUNT} resets to 0 and \texttt{LOCK\_STATE} becomes 1.
    \end{itemize}
    \subparagraph{Expectations of Test}
    \begin{center}
    \begin{tabular}{|c|p{10cm}|}
      \hline
      \textbf{Result} & \textbf{Conditions} \\
      \hline
      \textbf{PASS} &
        \begin{minipage}[t]{\linewidth}
        \begin{itemize}
          \item Door unlocks immediately upon entering the emergency PIN.
          \item \texttt{LOCKOUT\_COUNT} is reset to 0 and \texttt{LOCK\_STATE = 1} in Firestore.
          \item Mobile app shows “Emergency unlock used” notification within 5 s.
          \item Firestore contains a new \texttt{EMERGENCY\_UNLOCK} event entry. \\
        \end{itemize}
        \end{minipage} \\
      \hline
      \textbf{FAIL} & Any one of the PASS conditions is missing or incorrect. \\
      \hline
    \end{tabular}
    \end{center}
    
    \vspace{0.5em}
    
    \noindent\textbf{Reset after the test:}  
    Re-lock the door with a valid standard PIN to prepare for the next test.\\
    
\end{samepage}

% ---------------------------- Test 4 --------------------------------
\newpage
\begin{samepage}
    \subsection*{4. OTP Entry Test}
