\newpage
\section{Manufactured Product Testing}

\subsection*{Common Scenarios (Tests 1–10)}

\subsection*{1. Valid PIN Entry Test}
\subparagraph{Test Goals and Purpose}
\begin{itemize}
    \item Verify that a recognized 4-digit PIN unlocks the door without unnecessary delay.
    \item Check that the system logs this event as a successful access in Firestore.
\end{itemize}
\subparagraph{How We Test It}
\begin{itemize}
    \item Select a PIN known to be valid from the internal list.
    \item Enter it once and observe the system's response, both physical and digital.
    \item Look for updates to the LOCK\_STATE value in Firestore.
\end{itemize}
\subparagraph{Expectations of Test}
\begin{itemize}
    \item The bolt should retract within roughly half a second (this might vary slightly).
    \item LOCK\_STATE is set to 1 in Firestore upon unlock.
    \item There are no anomalies or unexpected entries in the logs.
\end{itemize}

\subsection*{2. Invalid PIN Lockout Test}
\subparagraph{Test Goals and Purpose}
\begin{itemize}
    \item Ensure the system prevents further PIN attempts after multiple consecutive failures.
    \item Confirm some form of lockout indication is presented, such as a light or alarm.
\end{itemize}
\subparagraph{How We Test It}
\begin{itemize}
    \item Enter three clearly incorrect PINs in under 30 seconds.
    \item Check the LOCKOUT\_COUNT value and observe for a visible or audible signal.
\end{itemize}
\subparagraph{Expectations of Test}
\begin{itemize}
    \item PIN entries are blocked for about 60 seconds following the third failure.
    \item Lockout state is visually or audibly indicated (could be LED or buzzer).
    \item Firestore logs each incorrect attempt, along with the lockout status.
\end{itemize}

\subsection*{3. Emergency PIN Test}
\subparagraph{Test Goals and Purpose}
\begin{itemize}
    \item Test whether a designated emergency PIN can override a lockout state.
    \item Ensure that using the emergency PIN resets the system's lockout logic.
\end{itemize}
\subparagraph{How We Test It}
\begin{itemize}
    \item Trigger a lockout by intentionally failing three PIN entries.
    \item Once lockout is active, input the emergency PIN.
    \item Monitor Firestore to verify an EMERGENCY\_UNLOCK entry is created.
\end{itemize}
\subparagraph{Expectations of Test}
\begin{itemize}
    \item The door should unlock immediately, without hesitation.
    \item Firestore logs the use of the emergency PIN distinctly.
    \item The system clears any active lockout status to resume normal operation.
\end{itemize}

\subsection*{4. OTP Entry Test}
\subparagraph{Test Goals and Purpose}
\begin{itemize}
    \item Confirm that an app-generated one-time password (OTP) grants access as expected.
    \item Ensure the OTP mechanism is time-bound and secure against reuse.
\end{itemize}
\subparagraph{How We Test It}
\begin{itemize}
    \item Request a 5-minute valid OTP from the app.
    \item Enter the OTP using the keypad.
    \item Monitor the system and database for changes.
\end{itemize}
\subparagraph{Expectations of Test}
\begin{itemize}
    \item Firestore captures the OTP with a clear expiry time.
    \item Valid OTPs allow immediate unlock.
    \item Expired or incorrect OTPs are properly rejected with no unlock attempt.
\end{itemize}

\subsection*{5. OTP Expiry Test}
\subparagraph{Test Goals and Purpose}
\begin{itemize}
    \item Ensure expired OTPs don’t grant access, even if they were once valid.
    \item Confirm that users are informed clearly when an OTP is no longer usable.
\end{itemize}
\subparagraph{How We Test It}
\begin{itemize}
    \item Generate an OTP and wait for the full 5-minute window to pass.
    \item Attempt to use the expired code via the keypad.
\end{itemize}
\subparagraph{Expectations of Test}
\begin{itemize}
    \item Lock remains unchanged and secure—no movement or unlock occurs.
    \item Firestore logs the expired OTP attempt explicitly.
    \item App and/or interface display an "OTP expired" notification.
\end{itemize}




























\newpage
\subsection*{Less Common Scenarios (Tests 31–40)}

\subsection*{31. Multi-factor Auth Proximity Test}
\subparagraph{Test Goals and Purpose}
\begin{itemize}
    \item Confirm the system requires both a valid PIN and Bluetooth proximity to unlock.
    \item Observe behavior if Bluetooth disconnects mid-process.
\end{itemize}
\subparagraph{How We Test It}
\begin{itemize}
    \item Start PIN entry and slowly move the paired phone out of Bluetooth range.
    \item Review logs to see how the system treats incomplete multi-factor input.
\end{itemize}
\subparagraph{Expectations of Test}
\begin{itemize}
    \item Door only unlocks when both factors are simultaneously validated.
    \item Firestore logs clearly show whether it was the PIN or BLE that failed.
\end{itemize}

\subsection*{32. Time-zone Mismatch Test}
\subparagraph{Test Goals and Purpose}
\begin{itemize}
    \item Ensure that timing-sensitive operations work properly despite device time mismatches.
    \item Verify that scheduled actions and OTPs remain aligned regardless of local time zones.
\end{itemize}
\subparagraph{How We Test It}
\begin{itemize}
    \item Set the phone to Pacific Time and the lock to Coordinated Universal Time.
    \item Generate and use an OTP, and observe if the mismatch causes issues.
\end{itemize}
\subparagraph{Expectations of Test}
\begin{itemize}
    \item OTPs still work during their intended validity window.
    \item Scheduled events (e.g., auto-lock) trigger based on synchronized absolute times.
\end{itemize}

