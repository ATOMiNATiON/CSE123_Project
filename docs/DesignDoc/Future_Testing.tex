\newpage
\begin{samepage}
    \section{Manufactured Product Testing}
    \subsection{Common Scenarios (Tests 1–20)}
    % ---------------------------- Test 1 --------------------------------
    \subsection*{1. Valid PIN Entry Test}
    \subparagraph{Test Goals and Purpose}
    \begin{itemize}
        \item Verify that a recognized 4-digit PIN unlocks the door without unnecessary delay.
        \item Check that the system logs this event as a successful access in Firestore.
    \end{itemize}
    \subparagraph{How We Test It}
    \begin{enumerate}
        \item Select a PIN known to be valid from the internal list.
        \item Enter the PIN once on the keypad.
        \item Observe the system's response, checking for physical bolt movement and UI changes on the mobile application.
        \item Look for updates to the (\texttt{LOCK\_STATE}) value in Firestore.
    \end{enumerate}
    
    \textbf{Expectations of Test}
    \begin{center}
    \begin{tabular}{|c|p{10cm}|}
      \hline
      \textbf{Result} & \textbf{Conditions} \\
      \hline
      PASS & Bolt retracts within \textless{}0.5\,s; \texttt{LOCK\_STATE}=1; Firestore log entry \emph{success}. \\
      \hline
      FAIL & Any deviation from the above timing, state, or logging behaviour. \\
      \hline
    \end{tabular}
    \end{center}
\end{samepage}


% ---------------------------- Test 2 --------------------------------
\newpage
\subsection*{2. Invalid PIN Lockout Test}
\subparagraph{Test Goals and Purpose}
\begin{itemize}
    \item Ensure the system prevents further PIN attempts after multiple consecutive failures.
    \item Lock should be shut down for a short expiration period.
    \item Confirm some form of lockout indication is presented, such as a light or warning beep.
    \item Owner's phoone should receive an alert about the failed attempts.
\end{itemize}
\subparagraph{How We Test It}
\begin{itemize}
    \item Enter three clearly incorrect PINs in under 30 seconds.
    \item Observe for a visible or audible signal.
    \item Confirm that \texttt{LOCKOUT\_COUNT} has risen and \texttt{LOCK\_STATE} is still 0 (locked).
\end{itemize}
\subparagraph{Expectations of Test}
\begin{center}
\begin{tabular}{|c|p{10cm}|}
  \hline
  \textbf{Result} & \textbf{Conditions} \\
  \hline
  \textbf{PASS} &
    \begin{minipage}[t]{\linewidth}
    \begin{itemize}
      \item Keypad flashes red and/or beeps immediately after the \textbf{third} wrong entry.
      \item No further PINs are accepted for the lockout period (default: 60 s).
      \item Mobile app log shows “3 invalid PIN attempts – lock is temporarily disabled” within 5 s.
      \item Firestore records \texttt{LOCKOUT\_COUNT = 3} and \texttt{LOCK\_STATE = 0}.\\
    \end{itemize}
    \end{minipage} \\ 
  \hline
  \textbf{FAIL} & Any one of the PASS conditions is missing or incorrect. \\ 
  \hline
\end{tabular}
\end{center}
\vspace{0.5em}

\noindent\textbf{Reset after the test:}  
Either wait for the lockout timer to expire or enter an admin PIN to clear the lockout so the next test starts from a clean state.\\


% ---------------------------- Test 3 --------------------------------
\newpage
\begin{samepage}
    \subsection*{3. Emergency PIN Test}
    \subparagraph{Test Goals and Purpose}
    \begin{itemize}
        \item Verify that a designated emergency PIN can override an active lockout state.
        \item Ensure that using the emergency PIN resets lockout counters and restores normal operation.
        \item Confirm the system logs the emergency unlock event.
    \end{itemize}
    \subparagraph{How We Test It}
    \begin{itemize}
        \item Trigger a lockout by intentionally inputting three incorrect PIN entries.
        \item Confirm after each incorrect PIN attempt (\texttt{LOCK\_STATE} = 1).
        \item Immediately enter the emergency PIN once the lockout state is active.
        \item Observe the lock mechanism (it should retract).
        \item Monitor Firestore to verify an EMERGENCY\_UNLOCK entry is created.
        \item Owner's phone should be notified and the mobile application should display that the emergency PIN was used.
        \item In Firestore, confirm that \texttt{LOCKOUT\_COUNT} resets to 0 and \texttt{LOCK\_STATE} becomes 1.
    \end{itemize}
    \subparagraph{Expectations of Test}
    \begin{center}
    \begin{tabular}{|c|p{10cm}|}
      \hline
      \textbf{Result} & \textbf{Conditions} \\
      \hline
      \textbf{PASS} &
        \begin{minipage}[t]{\linewidth}
        \begin{itemize}
          \item Door unlocks immediately upon entering the emergency PIN.
          \item \texttt{LOCKOUT\_COUNT} is reset to 0 and \texttt{LOCK\_STATE = 1} in Firestore.
          \item Mobile app shows “Emergency unlock used” notification within 5 s.
          \item Firestore contains a new \texttt{EMERGENCY\_UNLOCK} event entry. \\
        \end{itemize}
        \end{minipage} \\
      \hline
      \textbf{FAIL} & Any one of the PASS conditions is missing or incorrect. \\
      \hline
    \end{tabular}
    \end{center}
    
    \vspace{0.5em}
    
    \noindent\textbf{Reset after the test:}  
    Re-lock the door with a valid standard PIN to prepare for the next test.\\
    
\end{samepage}



\subsection*{4. One Time Pin Entry Test}
\subparagraph{Test Goals and Purpose}
\begin{itemize}
    \item Confirm that an app-generated one-time password (OTP) grants access as expected.
    \item Ensure the OTP mechanism is time-bound and secure against reuse.
\end{itemize}
\subparagraph{How We Test It}
\begin{itemize}
    \item Request a 5-minute valid OTP from the app.
    \item Enter the OTP using the keypad.
    \item Monitor the system and database for changes.
\end{itemize}
\subparagraph{Expectations of Test}
\begin{center}
    \begin{tabular}{|c|p{10cm}|}
      \hline
      \textbf{Result} & \textbf{Conditions} \\
      \hline
      \textbf{PASS} &
        \begin{minipage}[t]{\linewidth}
        \begin{itemize}
          \item Valid OTPs allow immediate unlock
          \item \texttt{LOCK\_STATE = 1} in Firestore.
          \item Door automatically locks itself after 10 s
          \item Mobile app shows "OTP unlock used” notification within 5 s.
          \item Firestore captures the OTP with a clear expiry time.
          \item Expired or incorrect OTPs are properly rejected with no unlock attempt. \\
        \end{itemize}
        \end{minipage} \\
      \hline
      \textbf{FAIL} & Any one of the PASS conditions is missing or incorrect. \\
      \hline
    \end{tabular}
    \end{center}

\subsection*{5. OTP Expiry Test}
\subparagraph{Test Goals and Purpose}
\begin{itemize}
    \item Ensure expired OTPs don’t grant access, even if they were once valid.
    \item Confirm that users are informed clearly when an OTP is no longer usable.
\end{itemize}
\subparagraph{How We Test It}
\begin{itemize}
    \item Generate an OTP and wait for the full 5-minute window to pass.
    \item Attempt to use the expired code via the keypad.
\end{itemize}
\subparagraph{Expectations of Test}
\begin{center}
    \begin{tabular}{|c|p{10cm}|}
      \hline
      \textbf{Result} & \textbf{Conditions} \\
      \hline
      \textbf{PASS} &
        \begin{minipage}[t]{\linewidth}
        \begin{itemize}
          \item Lock remains unchanged and secure—no movement or unlock occurs.
          \item Firestore logs the expired OTP attempt explicitly.
          \item App and/or interface display an "OTP expired" notification. \\
        \end{itemize}
        \end{minipage} \\
      \hline
      \textbf{FAIL} & Any one of the PASS conditions is missing or incorrect. \\
      \hline
    \end{tabular}
    \end{center}

\subsection*{6. Rapid Wrong PIN Alarm Test}
\subparagraph{Test Goals and Purpose}
\begin{itemize}
    \item Ensure multiple fast wrong PIN entries trigger audible alarm.
    \item Verify lock does not unlock after wrong entries.
    \item Check alarm stops after timeout or correct PIN.
\end{itemize}
\subparagraph{How We Test It}
\begin{itemize}
    \item Enter 5 wrong PINs within 10s.
    \item Observe alarm sound and LOCK\_STATE remains 0.
    \item Attempt correct PIN after alarm to confirm reset.
\end{itemize}
\subparagraph{Expectations of Test}
\begin{center}
    \begin{tabular}{|c|p{10cm}|}
      \hline
      \textbf{Result} & \textbf{Conditions} \\
      \hline
      \textbf{PASS} &
        \begin{minipage}[t]{\linewidth}
        \begin{itemize}
          \item Alarm beeps continuously for spec duration (e.g., 30s).
          \item Bolt remains locked (LOCK\_STATE = 0).
          \item Firestore logs each failed attempt and alarm event.
          \item Correct PIN after alarm silences alarm and unlocks door. \\
        \end{itemize}
        \end{minipage} \\
      \hline
      \textbf{FAIL} & Any one of the PASS conditions is missing or incorrect. \\
      \hline
    \end{tabular}
    \end{center}

\subsection*{7. Concurrent Unlock Command Test}
\subparagraph{Test Goals and Purpose}
\begin{itemize}
    \item Verify two family phones sending unlock at once doesn't confuse the system.
    \item Ensure only one actuation happens.
    \item Confirm both requests are logged.
\end{itemize}
\subparagraph{How We Test It}
\begin{itemize}
    \item From two phones, send unlock API calls within 100 ms.
    \item Watch bolt movement and LOCK\_STATE.
    \item Check Firestore for two event entries.
\end{itemize}
\subparagraph{Expectations of Test}
\begin{center}
    \begin{tabular}{|c|p{10cm}|}
      \hline
      \textbf{Result} & \textbf{Conditions} \\
      \hline
      \textbf{PASS} &
        \begin{minipage}[t]{\linewidth}
        \begin{itemize}
          \item Bolt retracts only once.
          \item LOCK\_STATE = 1 after first command.
          \item Both commands logged with timestamps.
          \item No errors or retries triggered. \\
        \end{itemize}
        \end{minipage} \\
      \hline
      \textbf{FAIL} & Any one of the PASS conditions is missing or incorrect. \\
      \hline
    \end{tabular}
    \end{center}

\subsection*{8. Offline Fallback PIN Test}
\subparagraph{Test Goals and Purpose}
\begin{itemize}
    \item Ensure local PIN entry still works if Wi-Fi drops.
    \item Verify lock uses cached PIN list.
    \item Confirm Firestore syncs later.
\end{itemize}
\subparagraph{How We Test It}
\begin{itemize}
    \item Disable Wi-Fi on ESP32, enter valid PIN.
    \item Observe bolt movement and local log.
    \item Re-enable Wi-Fi and check Firestore sync.
\end{itemize}
\subparagraph{Expectations of Test}

\begin{center}
    \begin{tabular}{|c|p{10cm}|}
      \hline
      \textbf{Result} & \textbf{Conditions} \\
      \hline
      \textbf{PASS} &
        \begin{minipage}[t]{\linewidth}
        \begin{itemize}
          \item Door unlocks locally (LOCK\_STATE = 1 locally).
          \item Event cached and later pushed to Firestore.
          \item No user-perceived delay in unlocking.
          \item Sync event marked with “offline” flag. \\
        \end{itemize}
        \end{minipage} \\
      \hline
      \textbf{FAIL} & Any one of the PASS conditions is missing or incorrect. \\
      \hline
    \end{tabular}
    \end{center}

\subsection*{9. Low-Battery Notification Test}
\subparagraph{Test Goals and Purpose}
\begin{itemize}
    \item Check notification when battery falls below 10 %.
    \item Ensure unlock still works at low battery.
    \item Verify UI warning persists until charge.
\end{itemize}
\subparagraph{How We Test It}
\begin{itemize}
    \item Drain battery to 9 %.
    \item Press unlock PIN and observe behavior.
    \item Check mobile app for low-battery alert.
\end{itemize}
\subparagraph{Expectations of Test}

\begin{center}
    \begin{tabular}{|c|p{10cm}|}
      \hline
      \textbf{Result} & \textbf{Conditions} \\
      \hline
      \textbf{PASS} &
        \begin{minipage}[t]{\linewidth}
        \begin{itemize}
          \item Door still unlocks (LOCK\_STATE = 1).
          \item App shows persistent “Low Battery” message.
          \item Firestore logs battery level event.
          \item Warning clears only when battery > 20\% \\
        \end{itemize}
        \end{minipage} \\
      \hline
      \textbf{FAIL} & Any one of the PASS conditions is missing or incorrect. \\
      \hline
    \end{tabular}
    \end{center}

\subsection*{10. Battery Fully Drained Test}
\subparagraph{Test Goals and Purpose}
\begin{itemize}
    \item Verify lock fails to actuate when battery is dead.
    \item Ensure system logs failure and alerts user.
    \item Confirm physical key still works.
\end{itemize}
\subparagraph{How We Test It}
\begin{itemize}
    \item Let battery drop to 0\%.
    \item Enter valid PIN and observe no bolt movement.
    \item Try physical key override.
\end{itemize}
\subparagraph{Expectations of Test}
\begin{center}
    \begin{tabular}{|c|p{10cm}|}
      \hline
      \textbf{Result} & \textbf{Conditions} \\
      \hline
      \textbf{PASS} &
        \begin{minipage}[t]{\linewidth}
        \begin{itemize}
          \item Bolt does not move on electronic command.
          \item Firestore logs “Battery Depleted” error.
          \item Physical key unlocks door.
          \item App advises “Replace Battery” notification. \\
        \end{itemize}
        \end{minipage} \\
      \hline
      \textbf{FAIL} & Any one of the PASS conditions is missing or incorrect. \\
      \hline
    \end{tabular}
    \end{center}

\subsection*{11. Power Surge Resilience Test}
\subparagraph{Test Goals and Purpose}
\begin{itemize}
    \item Ensure the ESP32 safely handles sudden voltage spikes or drops (brown-out conditions).
    \item Verify that the system resets safely without hardware damage.
    \item Confirm that the device automatically resumes correct operation after reset.
\end{itemize}
\subparagraph{How We Test It}
\begin{itemize}
    \item Connect the device to a programmable power supply.
    \item Apply a rapid voltage increase (e.g., spike to 4.5V) or decrease (e.g., drop below 3.0V) to simulate surge or brown-out.
    \item Observe if the ESP32 performs a brown-out reset.
    \item Monitor recovery behavior and system logs.
\end{itemize}
\subparagraph{Expectations of Test}
\begin{center}
    \begin{tabular}{|c|p{10cm}|}
      \hline
      \textbf{Result} & \textbf{Conditions} \\
      \hline
      \textbf{PASS} &
        \begin{minipage}[t]{\linewidth}
        \begin{itemize}
          \item ESP32 performs a brown-out reset when voltage is out of range.
          \item No permanent hardware damage occurs.
          \item Device reconnects to Wi-Fi and Firestore automatically.
          \item System resumes normal operation within 10 seconds of recovery. \\
        \end{itemize}
        \end{minipage} \\
      \hline
      \textbf{FAIL} & Any one of the PASS conditions is missing or incorrect. \\
      \hline
    \end{tabular}
    \end{center}

\subsection*{12. Mid-Unlock Reboot Recovery Test}
\subparagraph{Test Goals and Purpose}
\begin{itemize}
    \item Verify that the smart lock can recover from an unexpected reboot during the unlock cycle.
    \item Ensure no mechanical or software errors occur due to mid-cycle interruption.
    \item Confirm that the system safely resumes to a consistent and correct lock state.
\end{itemize}

\subparagraph{How We Test It}
\begin{itemize}
    \item Initiate an unlock operation by entering a valid PIN or triggering via Firestore.
    \item While the lock bolt is actuating (in motion), manually reboot the ESP32 (e.g., press reset button).
    \item After reboot, observe the behavior of the lock and system logs.
\end{itemize}

\subparagraph{Expectations of Test}
\begin{center}
    \begin{tabular}{|c|p{10cm}|}
      \hline
      \textbf{Result} & \textbf{Conditions} \\
      \hline
      \textbf{PASS} & 
        \begin{minipage}[t]{\linewidth}
        \begin{itemize}
          \item Device reboots without hardware fault.
          \item Lock resumes to the correct and safe final position (locked or unlocked).
          \item State in Firestore is updated accurately post-recovery.
          \item System logs include a reboot event and status report.
        \end{itemize}
        \end{minipage} \\
      \hline
      \textbf{FAIL} & Any one of the PASS conditions is missing or incorrect. \\
      \hline
    \end{tabular}
\end{center}

\subsection*{13. Duplicate Cloud Command Handling Test}
\subparagraph{Test Goals and Purpose}
\begin{itemize}
    \item Ensure that the smart lock ignores repeated unlock commands with the same intent.
    \item Prevent mechanical wear or electrical issues from unnecessary double-actuation.
    \item Confirm that the system properly logs and handles duplicate commands.
\end{itemize}

\subparagraph{How We Test It}
\begin{itemize}
    \item Send an unlock command via Firestore with a known state change (e.g., `isLocked: false`).
    \item Observe that the lock successfully unlocks on first command.
    \item Without changing the command value (keep `isLocked: false`), trigger a second Firestore write with the same content.
    \item Monitor system behavior, actuator activity, and system logs.
\end{itemize}

\subparagraph{Expectations of Test}
\begin{center}
    \begin{tabular}{|c|p{10cm}|}
      \hline
      \textbf{Result} & \textbf{Conditions} \\
      \hline
      \textbf{PASS} & 
        \begin{minipage}[t]{\linewidth}
        \begin{itemize}
          \item Lock actuates only once for the initial command.
          \item Second duplicate message is ignored safely.
          \item System logs indicate duplicate command was received and skipped.
          \item No mechanical movement or audible actuation occurs on duplicate.
        \end{itemize}
        \end{minipage} \\
      \hline
      \textbf{FAIL} & Any one of the PASS conditions is missing or incorrect. \\
      \hline
    \end{tabular}
\end{center}

\subsection*{14. Solenoid Jam Detection Test}
\subparagraph{Test Goals and Purpose}
\begin{itemize}
    \item Verify that the system detects when the solenoid fails to complete its movement due to physical obstruction.
    \item Ensure that the system does not enter an unsafe or unknown state.
    \item Confirm that the system times out the unlock operation and alerts the user appropriately.
\end{itemize}

\subparagraph{How We Test It}
\begin{itemize}
    \item Manually obstruct the solenoid or insert debris to simulate a jam in the latch.
    \item Send an unlock command via keypad or Firestore.
    \item Observe whether the solenoid attempts to actuate, fails, and triggers timeout logic.
    \item Check system logs, app notifications, and Firestore status.
\end{itemize}

\subparagraph{Expectations of Test}
\begin{center}
    \begin{tabular}{|c|p{10cm}|}
      \hline
      \textbf{Result} & \textbf{Conditions} \\
      \hline
      \textbf{PASS} & 
        \begin{minipage}[t]{\linewidth}
        \begin{itemize}
          \item Solenoid attempt is detected and terminated within a defined timeout window (e.g., 3 seconds).
          \item System logs a “Mechanical Jam Detected” or similar error.
          \item Firestore reflects failure or stuck status.
          \item App displays a warning to the user (e.g., “Unlock Failed – Check Lock Mechanism”).
        \end{itemize}
        \end{minipage} \\
      \hline
      \textbf{FAIL} & Any one of the PASS conditions is missing or incorrect. \\
      \hline
    \end{tabular}
\end{center}

\subsection*{15. Physical Key Override While Powered Test}
\subparagraph{Test Goals and Purpose}
\begin{itemize}
    \item Ensure that the lock can be manually overridden with a physical key even when the electronic system is powered.
    \item Confirm that the system detects the physical override and updates its internal state accordingly.
    \item Prevent any mechanical or electrical conflicts between the solenoid and manual actuation.
\end{itemize}

\subparagraph{How We Test It}
\begin{itemize}
    \item Power on the smart lock system normally.
    \item Use the physical key to manually unlock or lock the door.
    \item Observe system behavior (e.g., sensors, logs, Firestore sync).
    \item Attempt an electronic command (lock/unlock) immediately afterward to verify proper system alignment.
\end{itemize}

\subparagraph{Expectations of Test}
\begin{center}
    \begin{tabular}{|c|p{10cm}|}
      \hline
      \textbf{Result} & \textbf{Conditions} \\
      \hline
      \textbf{PASS} & 
        \begin{minipage}[t]{\linewidth}
        \begin{itemize}
          \item Manual override is mechanically smooth and does not interfere with solenoid.
          \item System detects state change via sensor (e.g., magnetic or position feedback).
          \item Firestore and app update to reflect new lock state.
          \item Subsequent electronic commands work normally without misalignment.
        \end{itemize}
        \end{minipage} \\
      \hline
      \textbf{FAIL} & Any one of the PASS conditions is missing or incorrect. \\
      \hline
    \end{tabular}
\end{center}

\subsection*{16. Hall-Sensor Noise Rejection Test}
\subparagraph{Test Goals and Purpose}
\begin{itemize}
    \item Verify that the system correctly interprets magnetic sensor signals despite electrical noise or transient spikes.
    \item Prevent false positives (e.g., reporting "locked" when bolt hasn't reached the lock position).
    \item Ensure the lock status reported to the app and Firestore is accurate and trustworthy.
\end{itemize}

\subparagraph{How We Test It}
\begin{itemize}
    \item Simulate magnetic noise or signal fluctuations near the Hall-effect sensor using a signal generator or nearby magnet source.
    \item Observe whether the system falsely detects a "locked" or "unlocked" state.
    \item Compare physical bolt position with system status shown in Firestore and app.
    \item Repeat under normal operation and edge cases (e.g., during solenoid movement).
\end{itemize}

\subparagraph{Expectations of Test}
\begin{center}
    \begin{tabular}{|c|p{10cm}|}
      \hline
      \textbf{Result} & \textbf{Conditions} \\
      \hline
      \textbf{PASS} & 
        \begin{minipage}[t]{\linewidth}
        \begin{itemize}
          \item System does not falsely report "locked" or "unlocked" due to magnetic interference.
          \item Lock state in Firestore matches actual bolt position.
          \item App accurately reflects the physical lock status.
          \item Optional: Sensor readings are filtered or debounced in software to ignore noise.
        \end{itemize}
        \end{minipage} \\
      \hline
      \textbf{FAIL} & Any one of the PASS conditions is missing or incorrect. \\
      \hline
    \end{tabular}
\end{center}

\subsection*{17. I²C Temperature Sensor Fault Tolerance Test}
\subparagraph{Test Goals and Purpose}
\begin{itemize}
    \item Verify the system handles brief communication errors or invalid data from the temperature sensor gracefully.
    \item Ensure no crashes, system hangs, or misleading readings are shown to the user.
    \item Confirm that erroneous data is filtered or flagged appropriately.
\end{itemize}

\subparagraph{How We Test It}
\begin{itemize}
    \item Emulate an I²C fault using a glitch generator, cable disturbance, or by temporarily disconnecting the sensor.
    \item Allow the sensor to resume normal operation after 1–2 cycles.
    \item Observe system logs, display/app data, and Firestore fields related to temperature.
    \item Confirm whether system handles retry logic or flags invalid reads.
\end{itemize}

\subparagraph{Expectations of Test}
\begin{center}
    \begin{tabular}{|c|p{10cm}|}
      \hline
      \textbf{Result} & \textbf{Conditions} \\
      \hline
      \textbf{PASS} & 
        \begin{minipage}[t]{\linewidth}
        \begin{itemize}
          \item System does not crash, hang, or misbehave when invalid data is received.
          \item Invalid temperature readings are ignored or marked as “error” in logs.
          \item Firestore/app either retains last valid value or displays "unavailable".
          \item Normal readings resume once sensor stabilizes.
        \end{itemize}
        \end{minipage} \\
      \hline
      \textbf{FAIL} & Any one of the PASS conditions is missing or incorrect. \\
      \hline
    \end{tabular}
\end{center}

\subsection*{18. RTC Drift Test After Offline Period}
\subparagraph{Test Goals and Purpose}
\begin{itemize}
    \item Measure the time drift of the ESP32’s RTC after prolonged offline operation (e.g., one week without network time synchronization).
    \item Assess whether the RTC accuracy meets system requirements for timestamping events and logs.
    \item Determine if corrective measures (e.g., periodic NTP sync) are necessary.
\end{itemize}

\subparagraph{How We Test It}
\begin{itemize}
    \item Synchronize the ESP32’s RTC to accurate time (e.g., via NTP).
    \item Disconnect the device from the network and keep powered for one week.
    \item After one week, compare the ESP32 RTC time against a reference clock (e.g., atomic clock, smartphone).
    \item Record the time difference (drift) in minutes.
\end{itemize}

\subparagraph{Expectations of Test}
\begin{center}
    \begin{tabular}{|c|p{10cm}|}
      \hline
      \textbf{Result} & \textbf{Conditions} \\
      \hline
      \textbf{PASS} & 
        \begin{minipage}[t]{\linewidth}
        \begin{itemize}
          \item RTC drift is less than or equal to 1 minute after one week offline.
          \item System logs reflect consistent and accurate timestamps.
          \item No significant impact on event timing or logging accuracy.
        \end{itemize}
        \end{minipage} \\
      \hline
      \textbf{FAIL} & Any one of the PASS conditions is missing or incorrect. \\
      \hline
    \end{tabular}
\end{center}

\subsection*{19. High Temperature Operation Test}
\subparagraph{Test Goals and Purpose}
\begin{itemize}
    \item Verify the smart lock’s ability to operate correctly when internal temperature exceeds 80 °C.
    \item Ensure system stability, sensor accuracy, and safe operation under elevated temperature conditions.
    \item Identify any thermal shutdown, degradation, or failure modes.
\end{itemize}

\subparagraph{How We Test It}
\begin{itemize}
    \item Place the smart lock in an environmental chamber or attic-like setup where temperature can be controlled.
    \item Gradually increase ambient temperature to exceed 80 °C inside the lock enclosure.
    \item Monitor system behavior, including lock/unlock commands, sensor readings, and logs.
    \item Record any faults, error messages, or abnormal operation.
\end{itemize}

\subparagraph{Expectations of Test}
\begin{center}
    \begin{tabular}{|c|p{10cm}|}
      \hline
      \textbf{Result} & \textbf{Conditions} \\
      \hline
      \textbf{PASS} & 
        \begin{minipage}[t]{\linewidth}
        \begin{itemize}
          \item Lock operates normally without unexpected resets or malfunctions.
          \item Temperature sensor accurately reports temperature exceeding 80 °C.
          \item System logs thermal condition and, if applicable, triggers thermal protection or warnings.
          \item No permanent damage or degradation to components.
        \end{itemize}
        \end{minipage} \\
      \hline
      \textbf{FAIL} & Any one of the PASS conditions is missing or incorrect. \\
      \hline
    \end{tabular}
\end{center}

\subsection*{20. Low Temperature Operation Test}
\subparagraph{Test Goals and Purpose}
\begin{itemize}
    \item Verify the smart lock operates correctly at low temperatures down to -10 °C.
    \item Ensure mechanical components (solenoid, bolt, physical key) function without freezing or sticking.
    \item Confirm electronic components, sensors, and battery performance remain stable.
\end{itemize}

\subparagraph{How We Test It}
\begin{itemize}
    \item Place the smart lock in a controlled environmental chamber set to -10 °C.
    \item Perform locking and unlocking operations electronically and via physical key.
    \item Monitor system responsiveness, sensor readings, battery voltage, and any error logs.
    \item Repeat test cycles to simulate typical usage during cold conditions.
\end{itemize}

\subparagraph{Expectations of Test}
\begin{center}
    \begin{tabular}{|c|p{10cm}|}
      \hline
      \textbf{Result} & \textbf{Conditions} \\
      \hline
      \textbf{PASS} & 
        \begin{minipage}[t]{\linewidth}
        \begin{itemize}
          \item Lock and unlock commands execute successfully without delay or mechanical failure.
          \item Sensors and electronic systems operate within specified parameters.
          \item Battery performance supports normal operation at low temperature.
          \item No freezing, sticking, or damage to mechanical components.
        \end{itemize}
        \end{minipage} \\
      \hline
      \textbf{FAIL} & Any one of the PASS conditions is missing or incorrect. \\
      \hline
    \end{tabular}
\end{center}

\subsection*{21. High Humidity and Condensation Test}
\subparagraph{Test Goals and Purpose}
\begin{itemize}
    \item Verify the smart lock’s PCB and electronics withstand prolonged exposure to high humidity and condensation.
    \item Ensure no corrosion, shorts, or malfunctions occur due to moisture.
    \item Confirm system stability and proper operation during and after moisture exposure.
\end{itemize}

\subparagraph{How We Test It}
\begin{itemize}
    \item Place the smart lock inside a humidity chamber set to high relative humidity (e.g., 90\% RH) at ambient temperature.
    \item Introduce temperature cycling to induce condensation on PCB surfaces.
    \item Operate the lock during and after exposure to monitor functionality.
    \item Inspect PCB visually for corrosion or moisture damage.
    \item Perform electrical continuity and insulation resistance tests post-exposure.
\end{itemize}

\subparagraph{Expectations of Test}
\begin{center}
    \begin{tabular}{|c|p{10cm}|}
      \hline
      \textbf{Result} & \textbf{Conditions} \\
      \hline
      \textbf{PASS} & 
        \begin{minipage}[t]{\linewidth}
        \begin{itemize}
          \item Smart lock operates normally during and after humidity exposure.
          \item No electrical shorts or intermittent failures occur.
          \item No visible corrosion or damage on PCB or components.
          \item Insulation resistance and continuity tests remain within specifications.
        \end{itemize}
        \end{minipage} \\
      \hline
      \textbf{FAIL} & Any one of the PASS conditions is missing or incorrect. \\
      \hline
    \end{tabular}
\end{center}

\subsection*{22. GPIO Pin Short to Ground Test}
\subparagraph{Test Goals and Purpose}
\begin{itemize}
    \item Verify system behavior and fault tolerance when a GPIO pin connected to the LED circuit is shorted to ground.
    \item Ensure no permanent damage occurs to the ESP32 or associated circuitry.
    \item Confirm the system continues normal operation despite LED failure.
\end{itemize}

\subparagraph{How We Test It}
\begin{itemize}
    \item Intentionally short the GPIO pin controlling the LED to ground.
    \item Power the system and observe behavior of the LED circuit and overall system.
    \item Monitor for signs of damage such as overheating, crashes, or erratic behavior.
    \item Check system logs and responses to LED control commands.
\end{itemize}

\subparagraph{Expectations of Test}
\begin{center}
    \begin{tabular}{|c|p{10cm}|}
      \hline
      \textbf{Result} & \textbf{Conditions} \\
      \hline
      \textbf{PASS} & 
        \begin{minipage}[t]{\linewidth}
        \begin{itemize}
          \item No damage to the ESP32 or LED circuitry.
          \item System continues to operate normally without resets or faults.
          \item LED fails to light as expected due to the short.
          \item System logs indicate LED malfunction or fault if applicable.
        \end{itemize}
        \end{minipage} \\
      \hline
      \textbf{FAIL} & Any one of the PASS conditions is missing or incorrect. \\
      \hline
    \end{tabular}
\end{center}

\subsection*{23. OTA Firmware Update Success Test}
\subparagraph{Test Goals and Purpose}
\begin{itemize}
    \item Confirm that the OTA firmware update process completes successfully on the first attempt.
    \item Ensure the smart lock remains operational and stable during and after the update.
    \item Verify that the new firmware version is correctly installed and running.
\end{itemize}

\subparagraph{How We Test It}
\begin{itemize}
    \item Initiate an OTA firmware update with a known stable firmware build.
    \item Monitor the update progress for any errors or interruptions.
    \item After completion, verify the firmware version and system functionality.
    \item Perform standard lock/unlock operations to confirm normal behavior.
\end{itemize}

\subparagraph{Expectations of Test}
\begin{center}
    \begin{tabular}{|c|p{10cm}|}
      \hline
      \textbf{Result} & \textbf{Conditions} \\
      \hline
      \textbf{PASS} & 
        \begin{minipage}[t]{\linewidth}
        \begin{itemize}
          \item OTA update completes without errors or retries.
          \item Smart lock boots into new firmware version successfully.
          \item Normal operation resumes immediately after update.
          \item No data loss or configuration resets occur.
        \end{itemize}
        \end{minipage} \\
      \hline
      \textbf{FAIL} & Any one of the PASS conditions is missing or incorrect. \\
      \hline
    \end{tabular}
\end{center}

\subsection*{24. OTA Update Failure and Rollback Test}
\subparagraph{Test Goals and Purpose}
\begin{itemize}
    \item Verify that the smart lock safely handles OTA update interruptions during mid-download.
    \item Confirm the system automatically rolls back to the prior stable firmware version.
    \item Ensure no corruption or bricking occurs, maintaining device operability.
\end{itemize}

\subparagraph{How We Test It}
\begin{itemize}
    \item Initiate an OTA firmware update.
    \item Simulate failure by interrupting the download mid-transfer (e.g., disconnect network or power).
    \item Power cycle the device and observe boot behavior.
    \item Verify the device boots the previous firmware version without errors.
    \item Test normal lock/unlock functionality after rollback.
\end{itemize}

\subparagraph{Expectations of Test}
\begin{center}
    \begin{tabular}{|c|p{10cm}|}
      \hline
      \textbf{Result} & \textbf{Conditions} \\
      \hline
      \textbf{PASS} & 
        \begin{minipage}[t]{\linewidth}
        \begin{itemize}
          \item Device detects incomplete or corrupted OTA update.
          \item Automatically rolls back to the previous firmware version.
          \item Device boots successfully without entering a bricked state.
          \item Lock functions correctly with prior firmware after rollback.
          \item System logs the failed update and rollback event.
        \end{itemize}
        \end{minipage} \\
      \hline
      \textbf{FAIL} & Any one of the PASS conditions is missing or incorrect. \\
      \hline
    \end{tabular}
\end{center}

\subsection*{25. API Rate Limiting and HTTP 429 Handling Test}
\subparagraph{Test Goals and Purpose}
\begin{itemize}
    \item Verify the system correctly handles HTTP 429 (Too Many Requests) responses from the cloud API.
    \item Ensure the smart lock firmware gracefully responds to rate limiting without crashing or excessive retries.
    \item Confirm that the system backs off and retries after appropriate delay.
    \item Prevent denial of service or degraded performance due to API flooding.
\end{itemize}

\subparagraph{How We Test It}
\begin{itemize}
    \item Simulate flooding the API with a high volume of status poll requests from the homeowner’s app or device.
    \item Monitor cloud API to confirm it returns HTTP 429 responses when rate limit exceeded.
    \item Observe smart lock’s firmware handling of HTTP 429 errors.
    \item Verify backoff and retry logic in firmware triggers correctly.
    \item Check logs for proper error handling and user notifications if applicable.
\end{itemize}

\subparagraph{Expectations of Test}
\begin{center}
    \begin{tabular}{|c|p{10cm}|}
      \hline
      \textbf{Result} & \textbf{Conditions} \\
      \hline
      \textbf{PASS} & 
        \begin{minipage}[t]{\linewidth}
        \begin{itemize}
          \item Smart lock detects HTTP 429 responses and throttles request rate accordingly.
          \item Firmware implements exponential backoff or delay before retrying.
          \item System maintains stability without crashes or excessive resource use.
          \item Logs reflect rate limiting events and retry attempts.
          \item User experience remains smooth with appropriate notifications if needed.
        \end{itemize}
        \end{minipage} \\
      \hline
      \textbf{FAIL} & Any one of the PASS conditions is missing or incorrect. \\
      \hline
    \end{tabular}
\end{center}


% End for Common Scenarios (Tests 1 - 20)

\newpage
\subsection{Less Common Scenarios (Tests 31–40)}

\subsection*{31. Multi-factor Auth Proximity Test}
\subparagraph{Test Goals and Purpose}
\begin{itemize}
    \item Confirm the system requires both a valid PIN and Bluetooth proximity to unlock.
    \item Observe behavior if Bluetooth disconnects mid-process.
\end{itemize}
\subparagraph{How We Test It}
\begin{itemize}
    \item Start PIN entry and slowly move the paired phone out of Bluetooth range.
    \item Review logs to see how the system treats incomplete multi-factor input.
\end{itemize}
\subparagraph{Expectations of Test}
\begin{itemize}
    \item Door only unlocks when both factors are simultaneously validated.
    \item Firestore logs clearly show whether it was the PIN or BLE that failed.
\end{itemize}

\subsection*{32. Time-zone Mismatch Test}
\subparagraph{Test Goals and Purpose}
\begin{itemize}
    \item Ensure that timing-sensitive operations work properly despite device time mismatches.
    \item Verify that scheduled actions and OTPs remain aligned regardless of local time zones.
\end{itemize}
\subparagraph{How We Test It}
\begin{itemize}
    \item Set the phone to Pacific Time and the lock to Coordinated Universal Time.
    \item Generate and use an OTP, and observe if the mismatch causes issues.
\end{itemize}
\subparagraph{Expectations of Test}
\begin{itemize}
    \item OTPs still work during their intended validity window.
    \item Scheduled events (e.g., auto-lock) trigger based on synchronized absolute times.
\end{itemize}

\subsection*{33. DST Transition Auto-lock Test}
\subparagraph{Test Goals and Purpose}
\begin{itemize}
    \item Confirm that daylight saving time transitions don't disrupt automated locking.
\end{itemize}
\subparagraph{How We Test It}
\begin{itemize}
    \item Schedule an auto-lock event for 2:00 AM on the day of the DST shift.
    \item Observe actual lock behavior before, during, and after the transition.
\end{itemize}
\subparagraph{Expectations of Test}
\begin{itemize}
    \item Only one auto-lock event occurs, even with the clock shift.
    \item No skipped or duplicated scheduling happens.
\end{itemize}

\subsection*{34. Admin vs. Guest Role Change Test}
\subparagraph{Test Goals and Purpose}
\begin{itemize}
    \item Validate that role-based access control takes effect immediately after role changes.
\end{itemize}
\subparagraph{How We Test It}
\begin{itemize}
    \item Change a user's role (e.g., from guest to admin) directly in Firestore.
    \item Attempt access under both roles shortly after the change.
\end{itemize}
\subparagraph{Expectations of Test}
\begin{itemize}
    \item The new role is honored without requiring system restart or delay.
    \item Access permissions align with the updated role instantly.
\end{itemize}

\subsection*{35. BLE Range Boundary Test}
\subparagraph{Test Goals and Purpose}
\begin{itemize}
    \item Determine the Bluetooth unlocking behavior at different physical distances.
\end{itemize}
\subparagraph{How We Test It}
\begin{itemize}
    \item Gradually move the phone away from the lock in 0.5-meter steps, from 1 m up to 6 m.
    \item Test unlocking at each position and note the results.
\end{itemize}
\subparagraph{Expectations of Test}
\begin{itemize}
    \item Unlocks reliably at closer distances.
    \item Fails gracefully beyond Bluetooth’s effective range.
\end{itemize}










































\newpage
\subsection{Rare Scenarios (Tests 61–65)}

\subsection*{61. Cosmic Ray Bit-Flip Test}
\subparagraph{Test Goals and Purpose}
\begin{itemize}
    \item Evaluate the system’s resilience to random memory errors, such as those caused by cosmic rays.
\end{itemize}
\subparagraph{How We Test It}
\begin{itemize}
    \item Use a simulator to inject random bit flips into RAM and flash every 10 seconds.
    \item Observe system behavior over an extended period.
\end{itemize}
\subparagraph{Expectations of Test}
\begin{itemize}
    \item System detects and handles bit errors—either by correcting or isolating them.
    \item No persistent crashes or loss of functionality.
\end{itemize}

\subsection*{62. Lightning Surge Test}
\subparagraph{Test Goals and Purpose}
\begin{itemize}
    \item Assess whether the hardware can tolerate sudden electrical surges.
\end{itemize}
\subparagraph{How We Test It}
\begin{itemize}
    \item Apply a 1-kV spike to the 12V input for a 1 microsecond duration using a surge generator.
\end{itemize}
\subparagraph{Expectations of Test}
\begin{itemize}
    \item No hardware is permanently damaged.
    \item System recovers and operates normally after the surge event.
\end{itemize}

\subsection*{63. Seismic Vibration Test}
\subparagraph{Test Goals and Purpose}
\begin{itemize}
    \item Verify the lock’s mechanical and electronic components withstand prolonged shaking.
\end{itemize}
\subparagraph{How We Test It}
\begin{itemize}
    \item Subject the device to a vibration range of 5–50 Hz at 0.5 g for 10 minutes.
    \item Check for loosening, noise, or malfunction.
\end{itemize}
\subparagraph{Expectations of Test}
\begin{itemize}
    \item All parts remain in place and fully operational.
    \item Sensors continue to function accurately post-test.
\end{itemize}

\subsection*{64. Extreme Cold Operation Test}
\subparagraph{Test Goals and Purpose}
\begin{itemize}
    \item Confirm functionality in severe cold conditions down to -40 °C.
\end{itemize}
\subparagraph{How We Test It}
\begin{itemize}
    \item Place the lock in a chamber set to -40 °C for 2 hours.
    \item Perform 10 complete lock/unlock cycles.
\end{itemize}
\subparagraph{Expectations of Test}
\begin{itemize}
    \item Mechanism operates without freezing or delay.
    \item No cracking or material degradation.
    \item Battery and electronics remain reliable.
\end{itemize}

\subsection*{65. Extreme Heat Operation Test}
\subparagraph{Test Goals and Purpose}
\begin{itemize}
    \item Test device reliability when exposed to 85 °C for extended periods.
\end{itemize}
\subparagraph{How We Test It}
\begin{itemize}
    \item Place the lock in a heat chamber at 85 °C.
    \item Execute a lock/unlock cycle every 5 minutes over 2 hours.
\end{itemize}
\subparagraph{Expectations of Test}
\begin{itemize}
    \item No thermal shutdown or malfunctions.
    \item MCU and storage remain intact and responsive.
\end{itemize}

