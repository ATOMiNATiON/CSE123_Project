\newpage
\begin{samepage}
    \section{Manufactured Product Testing}
    \subsection{Common Scenarios (Tests 1–10)}
    % ---------------------------- Test 1 --------------------------------
    \subsection*{1. Valid PIN Entry Test}
    \subparagraph{Test Goals and Purpose}
    \begin{itemize}
        \item Verify that a recognized 4-digit PIN unlocks the door without unnecessary delay.
        \item Check that the system logs this event as a successful access in Firestore.
    \end{itemize}
    \subparagraph{How We Test It}
    \begin{enumerate}
        \item Select a PIN known to be valid from the internal list.
        \item Enter the PIN once on the keypad.
        \item Observe the system's response, checking for physical bolt movement and UI changes on the mobile application.
        \item Look for updates to the (\texttt{LOCK\_STATE}) value in Firestore.
    \end{enumerate}
    
    \textbf{Expectations of Test}
    \begin{center}
    \begin{tabular}{|c|p{10cm}|}
      \hline
      \textbf{Result} & \textbf{Conditions} \\
      \hline
      PASS & Bolt retracts within \textless{}0.5\,s; \texttt{LOCK\_STATE}=1; Firestore log entry \emph{success}. \\
      \hline
      FAIL & Any deviation from the above timing, state, or logging behaviour. \\
      \hline
    \end{tabular}
    \end{center}
\end{samepage}


% ---------------------------- Test 2 --------------------------------
\newpage
\subsection*{2. Invalid PIN Lockout Test}
\subparagraph{Test Goals and Purpose}
\begin{itemize}
    \item Ensure the system prevents further PIN attempts after multiple consecutive failures.
    \item Lock should be shut down for a short expiration period.
    \item Confirm some form of lockout indication is presented, such as a light or warning beep.
    \item Owner's phoone should receive an alert about the failed attempts.
\end{itemize}
\subparagraph{How We Test It}
\begin{itemize}
    \item Enter three clearly incorrect PINs in under 30 seconds.
    \item Observe for a visible or audible signal.
    \item Confirm that \texttt{LOCKOUT\_COUNT} has risen and \texttt{LOCK\_STATE} is still 0 (locked).
\end{itemize}
\subparagraph{Expectations of Test}
\begin{center}
\begin{tabular}{|c|p{10cm}|}
  \hline
  \textbf{Result} & \textbf{Conditions} \\
  \hline
  \textbf{PASS} &
    \begin{minipage}[t]{\linewidth}
    \begin{itemize}
      \item Keypad flashes red and/or beeps immediately after the \textbf{third} wrong entry.
      \item No further PINs are accepted for the lockout period (default: 60 s).
      \item Mobile app log shows “3 invalid PIN attempts – lock is temporarily disabled” within 5 s.
      \item Firestore records \texttt{LOCKOUT\_COUNT = 3} and \texttt{LOCK\_STATE = 0}.\\
    \end{itemize}
    \end{minipage} \\ 
  \hline
  \textbf{FAIL} & Any one of the PASS conditions is missing or incorrect. \\ 
  \hline
\end{tabular}
\end{center}
\vspace{0.5em}

\noindent\textbf{Reset after the test:}  
Either wait for the lockout timer to expire or enter an admin PIN to clear the lockout so the next test starts from a clean state.\\


% ---------------------------- Test 3 --------------------------------
\newpage
\begin{samepage}
    \subsection*{3. Emergency PIN Test}
    \subparagraph{Test Goals and Purpose}
    \begin{itemize}
        \item Verify that a designated emergency PIN can override an active lockout state.
        \item Ensure that using the emergency PIN resets lockout counters and restores normal operation.
        \item Confirm the system logs the emergency unlock event.
    \end{itemize}
    \subparagraph{How We Test It}
    \begin{itemize}
        \item Trigger a lockout by intentionally inputting three incorrect PIN entries.
        \item Confirm after each incorrect PIN attempt (\texttt{LOCK\_STATE} = 1).
        \item Immediately enter the emergency PIN once the lockout state is active.
        \item Observe the lock mechanism (it should retract).
        \item Monitor Firestore to verify an EMERGENCY\_UNLOCK entry is created.
        \item Owner's phone should be notified and the mobile application should display that the emergency PIN was used.
        \item In Firestore, confirm that \texttt{LOCKOUT\_COUNT} resets to 0 and \texttt{LOCK\_STATE} becomes 1.
    \end{itemize}
    \subparagraph{Expectations of Test}
    \begin{center}
    \begin{tabular}{|c|p{10cm}|}
      \hline
      \textbf{Result} & \textbf{Conditions} \\
      \hline
      \textbf{PASS} &
        \begin{minipage}[t]{\linewidth}
        \begin{itemize}
          \item Door unlocks immediately upon entering the emergency PIN.
          \item \texttt{LOCKOUT\_COUNT} is reset to 0 and \texttt{LOCK\_STATE = 1} in Firestore.
          \item Mobile app shows “Emergency unlock used” notification within 5 s.
          \item Firestore contains a new \texttt{EMERGENCY\_UNLOCK} event entry. \\
        \end{itemize}
        \end{minipage} \\
      \hline
      \textbf{FAIL} & Any one of the PASS conditions is missing or incorrect. \\
      \hline
    \end{tabular}
    \end{center}
    
    \vspace{0.5em}
    
    \noindent\textbf{Reset after the test:}  
    Re-lock the door with a valid standard PIN to prepare for the next test.\\
    
\end{samepage}



\subsection*{4. One Time Pin Entry Test}
\subparagraph{Test Goals and Purpose}
\begin{itemize}
    \item Confirm that an app-generated one-time password (OTP) grants access as expected.
    \item Ensure the OTP mechanism is time-bound and secure against reuse.
\end{itemize}
\subparagraph{How We Test It}
\begin{itemize}
    \item Request a 5-minute valid OTP from the app.
    \item Enter the OTP using the keypad.
    \item Monitor the system and database for changes.
\end{itemize}
\subparagraph{Expectations of Test}
\begin{center}
    \begin{tabular}{|c|p{10cm}|}
      \hline
      \textbf{Result} & \textbf{Conditions} \\
      \hline
      \textbf{PASS} &
        \begin{minipage}[t]{\linewidth}
        \begin{itemize}
          \item Valid OTPs allow immediate unlock
          \item \texttt{LOCK\_STATE = 1} in Firestore.
          \item Door automatically locks itself after 10 s
          \item Mobile app shows "OTP unlock used” notification within 5 s.
          \item Firestore captures the OTP with a clear expiry time.
          \item Expired or incorrect OTPs are properly rejected with no unlock attempt. \\
        \end{itemize}
        \end{minipage} \\
      \hline
      \textbf{FAIL} & Any one of the PASS conditions is missing or incorrect. \\
      \hline
    \end{tabular}
    \end{center}

\subsection*{5. OTP Expiry Test}
\subparagraph{Test Goals and Purpose}
\begin{itemize}
    \item Ensure expired OTPs don’t grant access, even if they were once valid.
    \item Confirm that users are informed clearly when an OTP is no longer usable.
\end{itemize}
\subparagraph{How We Test It}
\begin{itemize}
    \item Generate an OTP and wait for the full 5-minute window to pass.
    \item Attempt to use the expired code via the keypad.
\end{itemize}
\subparagraph{Expectations of Test}
\begin{center}
    \begin{tabular}{|c|p{10cm}|}
      \hline
      \textbf{Result} & \textbf{Conditions} \\
      \hline
      \textbf{PASS} &
        \begin{minipage}[t]{\linewidth}
        \begin{itemize}
          \item Lock remains unchanged and secure—no movement or unlock occurs.
          \item Firestore logs the expired OTP attempt explicitly.
          \item App and/or interface display an "OTP expired" notification. \\
        \end{itemize}
        \end{minipage} \\
      \hline
      \textbf{FAIL} & Any one of the PASS conditions is missing or incorrect. \\
      \hline
    \end{tabular}
    \end{center}

\subsection*{6. Rapid Wrong PIN Alarm Test}
\subparagraph{Test Goals and Purpose}
\begin{itemize}
    \item Ensure multiple fast wrong PIN entries trigger audible alarm.
    \item Verify lock does not unlock after wrong entries.
    \item Check alarm stops after timeout or correct PIN.
\end{itemize}
\subparagraph{How We Test It}
\begin{itemize}
    \item Enter 5 wrong PINs within 10s.
    \item Observe alarm sound and LOCK\_STATE remains 0.
    \item Attempt correct PIN after alarm to confirm reset.
\end{itemize}
\subparagraph{Expectations of Test}
\begin{center}
    \begin{tabular}{|c|p{10cm}|}
      \hline
      \textbf{Result} & \textbf{Conditions} \\
      \hline
      \textbf{PASS} &
        \begin{minipage}[t]{\linewidth}
        \begin{itemize}
          \item Alarm beeps continuously for spec duration (e.g., 30s).
          \item Bolt remains locked (LOCK\_STATE = 0).
          \item Firestore logs each failed attempt and alarm event.
          \item Correct PIN after alarm silences alarm and unlocks door. \\
        \end{itemize}
        \end{minipage} \\
      \hline
      \textbf{FAIL} & Any one of the PASS conditions is missing or incorrect. \\
      \hline
    \end{tabular}
    \end{center}

\subsection*{7. Concurrent Unlock Command Test}
\subparagraph{Test Goals and Purpose}
\begin{itemize}
    \item Verify two family phones sending unlock at once doesn't confuse the system.
    \item Ensure only one actuation happens.
    \item Confirm both requests are logged.
\end{itemize}
\subparagraph{How We Test It}
\begin{itemize}
    \item From two phones, send unlock API calls within 100 ms.
    \item Watch bolt movement and LOCK\_STATE.
    \item Check Firestore for two event entries.
\end{itemize}
\subparagraph{Expectations of Test}
\begin{center}
    \begin{tabular}{|c|p{10cm}|}
      \hline
      \textbf{Result} & \textbf{Conditions} \\
      \hline
      \textbf{PASS} &
        \begin{minipage}[t]{\linewidth}
        \begin{itemize}
          \item Bolt retracts only once.
          \item LOCK\_STATE = 1 after first command.
          \item Both commands logged with timestamps.
          \item No errors or retries triggered. \\
        \end{itemize}
        \end{minipage} \\
      \hline
      \textbf{FAIL} & Any one of the PASS conditions is missing or incorrect. \\
      \hline
    \end{tabular}
    \end{center}

\subsection*{8. Offline Fallback PIN Test}
\subparagraph{Test Goals and Purpose}
\begin{itemize}
    \item Ensure local PIN entry still works if Wi-Fi drops.
    \item Verify lock uses cached PIN list.
    \item Confirm Firestore syncs later.
\end{itemize}
\subparagraph{How We Test It}
\begin{itemize}
    \item Disable Wi-Fi on ESP32, enter valid PIN.
    \item Observe bolt movement and local log.
    \item Re-enable Wi-Fi and check Firestore sync.
\end{itemize}
\subparagraph{Expectations of Test}
\begin{itemize}
    \item Door unlocks locally (LOCK\_STATE = 1 locally).
    \item Event cached and later pushed to Firestore.
    \item No user-perceived delay in unlocking.
    \item Sync event marked with “offline” flag.
\end{itemize}

\subsection*{9. Low-Battery Notification Test}
\subparagraph{Test Goals and Purpose}
\begin{itemize}
    \item Check notification when battery falls below 10 %.
    \item Ensure unlock still works at low battery.
    \item Verify UI warning persists until charge.
\end{itemize}
\subparagraph{How We Test It}
\begin{itemize}
    \item Drain battery to 9 %.
    \item Press unlock PIN and observe behavior.
    \item Check mobile app for low-battery alert.
\end{itemize}
\subparagraph{Expectations of Test}
\begin{itemize}
    \item Door still unlocks (LOCK\_STATE = 1).
    \item App shows persistent “Low Battery” message.
    \item Firestore logs battery level event.
    \item Warning clears only when battery > 20 %.
\end{itemize}

\subsection*{10. Battery Fully Drained Test}
\subparagraph{Test Goals and Purpose}
\begin{itemize}
    \item Verify lock fails to actuate when battery is dead.
    \item Ensure system logs failure and alerts user.
    \item Confirm physical key still works.
\end{itemize}
\subparagraph{How We Test It}
\begin{itemize}
    \item Let battery drop to 0\%.
    \item Enter valid PIN and observe no bolt movement.
    \item Try physical key override.
\end{itemize}
\subparagraph{Expectations of Test}
\begin{itemize}
    \item Bolt does not move on electronic command.
    \item Firestore logs “Battery Depleted” error.
    \item Physical key unlocks door.
    \item App advises “Replace Battery” notification.
\end{itemize}


\subsection*{11. Battery Fully Drained Test}
\subparagraph{Test Goals and Purpose}
\begin{itemize}
    \item Verify lock fails to actuate when battery is dead.
    \item Ensure system logs failure and alerts user.
    \item Confirm physical key still works.
\end{itemize}
\subparagraph{How We Test It}
\begin{itemize}
    \item Let battery drop to 0 \%.
    \item Enter valid PIN and observe no bolt movement.
    \item Try physical key override.
\end{itemize}
\subparagraph{Expectations of Test}
\begin{itemize}
    \item Bolt does not move on electronic command.
    \item Firestore logs “Battery Depleted” error.
    \item Physical key unlocks door.
    \item App advises “Replace Battery” notification.
\end{itemize}

























\newpage
\subsection{Less Common Scenarios (Tests 31–40)}

\subsection*{31. Multi-factor Auth Proximity Test}
\subparagraph{Test Goals and Purpose}
\begin{itemize}
    \item Confirm the system requires both a valid PIN and Bluetooth proximity to unlock.
    \item Observe behavior if Bluetooth disconnects mid-process.
\end{itemize}
\subparagraph{How We Test It}
\begin{itemize}
    \item Start PIN entry and slowly move the paired phone out of Bluetooth range.
    \item Review logs to see how the system treats incomplete multi-factor input.
\end{itemize}
\subparagraph{Expectations of Test}
\begin{itemize}
    \item Door only unlocks when both factors are simultaneously validated.
    \item Firestore logs clearly show whether it was the PIN or BLE that failed.
\end{itemize}

\subsection*{32. Time-zone Mismatch Test}
\subparagraph{Test Goals and Purpose}
\begin{itemize}
    \item Ensure that timing-sensitive operations work properly despite device time mismatches.
    \item Verify that scheduled actions and OTPs remain aligned regardless of local time zones.
\end{itemize}
\subparagraph{How We Test It}
\begin{itemize}
    \item Set the phone to Pacific Time and the lock to Coordinated Universal Time.
    \item Generate and use an OTP, and observe if the mismatch causes issues.
\end{itemize}
\subparagraph{Expectations of Test}
\begin{itemize}
    \item OTPs still work during their intended validity window.
    \item Scheduled events (e.g., auto-lock) trigger based on synchronized absolute times.
\end{itemize}

\subsection*{33. DST Transition Auto-lock Test}
\subparagraph{Test Goals and Purpose}
\begin{itemize}
    \item Confirm that daylight saving time transitions don't disrupt automated locking.
\end{itemize}
\subparagraph{How We Test It}
\begin{itemize}
    \item Schedule an auto-lock event for 2:00 AM on the day of the DST shift.
    \item Observe actual lock behavior before, during, and after the transition.
\end{itemize}
\subparagraph{Expectations of Test}
\begin{itemize}
    \item Only one auto-lock event occurs, even with the clock shift.
    \item No skipped or duplicated scheduling happens.
\end{itemize}

\subsection*{34. Admin vs. Guest Role Change Test}
\subparagraph{Test Goals and Purpose}
\begin{itemize}
    \item Validate that role-based access control takes effect immediately after role changes.
\end{itemize}
\subparagraph{How We Test It}
\begin{itemize}
    \item Change a user's role (e.g., from guest to admin) directly in Firestore.
    \item Attempt access under both roles shortly after the change.
\end{itemize}
\subparagraph{Expectations of Test}
\begin{itemize}
    \item The new role is honored without requiring system restart or delay.
    \item Access permissions align with the updated role instantly.
\end{itemize}

\subsection*{35. BLE Range Boundary Test}
\subparagraph{Test Goals and Purpose}
\begin{itemize}
    \item Determine the Bluetooth unlocking behavior at different physical distances.
\end{itemize}
\subparagraph{How We Test It}
\begin{itemize}
    \item Gradually move the phone away from the lock in 0.5-meter steps, from 1 m up to 6 m.
    \item Test unlocking at each position and note the results.
\end{itemize}
\subparagraph{Expectations of Test}
\begin{itemize}
    \item Unlocks reliably at closer distances.
    \item Fails gracefully beyond Bluetooth’s effective range.
\end{itemize}










































\newpage
\subsection{Rare Scenarios (Tests 61–65)}

\subsection*{61. Cosmic Ray Bit-Flip Test}
\subparagraph{Test Goals and Purpose}
\begin{itemize}
    \item Evaluate the system’s resilience to random memory errors, such as those caused by cosmic rays.
\end{itemize}
\subparagraph{How We Test It}
\begin{itemize}
    \item Use a simulator to inject random bit flips into RAM and flash every 10 seconds.
    \item Observe system behavior over an extended period.
\end{itemize}
\subparagraph{Expectations of Test}
\begin{itemize}
    \item System detects and handles bit errors—either by correcting or isolating them.
    \item No persistent crashes or loss of functionality.
\end{itemize}

\subsection*{62. Lightning Surge Test}
\subparagraph{Test Goals and Purpose}
\begin{itemize}
    \item Assess whether the hardware can tolerate sudden electrical surges.
\end{itemize}
\subparagraph{How We Test It}
\begin{itemize}
    \item Apply a 1-kV spike to the 12V input for a 1 microsecond duration using a surge generator.
\end{itemize}
\subparagraph{Expectations of Test}
\begin{itemize}
    \item No hardware is permanently damaged.
    \item System recovers and operates normally after the surge event.
\end{itemize}

\subsection*{63. Seismic Vibration Test}
\subparagraph{Test Goals and Purpose}
\begin{itemize}
    \item Verify the lock’s mechanical and electronic components withstand prolonged shaking.
\end{itemize}
\subparagraph{How We Test It}
\begin{itemize}
    \item Subject the device to a vibration range of 5–50 Hz at 0.5 g for 10 minutes.
    \item Check for loosening, noise, or malfunction.
\end{itemize}
\subparagraph{Expectations of Test}
\begin{itemize}
    \item All parts remain in place and fully operational.
    \item Sensors continue to function accurately post-test.
\end{itemize}

\subsection*{64. Extreme Cold Operation Test}
\subparagraph{Test Goals and Purpose}
\begin{itemize}
    \item Confirm functionality in severe cold conditions down to -40 °C.
\end{itemize}
\subparagraph{How We Test It}
\begin{itemize}
    \item Place the lock in a chamber set to -40 °C for 2 hours.
    \item Perform 10 complete lock/unlock cycles.
\end{itemize}
\subparagraph{Expectations of Test}
\begin{itemize}
    \item Mechanism operates without freezing or delay.
    \item No cracking or material degradation.
    \item Battery and electronics remain reliable.
\end{itemize}

\subsection*{65. Extreme Heat Operation Test}
\subparagraph{Test Goals and Purpose}
\begin{itemize}
    \item Test device reliability when exposed to 85 °C for extended periods.
\end{itemize}
\subparagraph{How We Test It}
\begin{itemize}
    \item Place the lock in a heat chamber at 85 °C.
    \item Execute a lock/unlock cycle every 5 minutes over 2 hours.
\end{itemize}
\subparagraph{Expectations of Test}
\begin{itemize}
    \item No thermal shutdown or malfunctions.
    \item MCU and storage remain intact and responsive.
\end{itemize}

