

\subsection{Introduction}
This document outlines the final design of the Smartlock system, detailing its major components and their interactions. The system leverages cloud infrastructure, secure authentication, wireless connectivity, and a robust data model to provide secure access.

\subsection{Cloud Infrastructure (AWS)}
The Smartlock system utilizes Amazon Web Services (AWS) as its cloud service. We will use AWS for the following:
\begin{itemize}
    \item Each SmartLock owner will have a unique AWS account and will have access to any other locks purchased through their personal cloud service. 
    \item No personal information will be stored on the cloud. The only information stored will be the lock(s) ID(s), the owner ID, the PIN codes, a history log, and other user IDs that the owner authorizes. Nothing else will be stored on the cloud.
\end{itemize}
AWS makes HTTPS easy and secure to use. First, you get a website certificate through AWS Certificate Manager, which proves your site is who it says it is and automatically renews itself. You then attach that certificate to our app. When someone visits our app, their iOS device checks the certificate and, if it’s valid, both sides agree on a secret key to encrypt all data sent back and forth. Behind the scenes, AWS uses its identity controls to keep your certificates safe and logs everything so you can see what’s happening. This way, you can trust that your login data and pin codes are secure.

\subsubsection*{Benefits of AWS Compared to Other Cloud Infrastructures For Smartlock}

AWS offers several advantages over other cloud solutions such as PostgreSQL or Google Firestore:
\begin{itemize}
    \item \textbf{Comprehensive Service Integration:} AWS provides a wide range of services (compute, storage, authentication, IoT, etc.) that work seamlessly together.
    \item \textbf{Scalability and Reliability:} AWS infrastructure is designed for automatic scaling to handle varying workloads, and in our case multiple users for multiple locks.
    \item \textbf{Managed Security:} Built-in https security features.
    \item \textbf{Automated Certificate Management:} AWS Certificate Manager simplifies HTTPS setup and renewal, reducing operational overhead.
    \item \textbf{Unified Monitoring and Logging:} Built-in tools for monitoring and logging data from app or smartlock.
    \item \textbf{Flexible Pricing Models:} Pay-as-you-go and reserved pricing options to optimize costs.
\end{itemize}

\subsection{PIN Code Access}
Each smartlock supports PIN code access for users. The PIN code system details these features.

\subsubsection*{One-Time PIN Codes}
\begin{itemize}
    \item One-time Pin Codes be generated on the app and sent to the cloud.
    \item Transmitted securely to the lock device through AWS IoT services.
    \item The lock verifies the code against the cloud database.
    \item One-time PIN codes are valid for a single use and expire after a set time period (e.g., 5-60 minutes).
    \item After the code is used, it is blacklisted in the cloud, app, and lock and cannot be reused.
    \item After code is used, time of use is logged in the cloud database.
\end{itemize}

\subsubsection*{Emergency PIN Codes}
\begin{itemize}
    \item Access to emergency PIN codes is restricted to the owner and authorized users, and will have an extra layer of security to access these pins.
    \item The codes are generated on app and sent to the cloud upon Wifi connection.
    \item Extra emergency PIN codes can be generated by the owner and authorized users
    \item After an emergency PIN code is used, that pin will be blacklisted from the app, cloud, and lock.
    \item All emergency PIN codes are refreshed upon WiFi reconnection. 
    \item Pins are validated locally to allow or deny access. Pin access is can be used even without internet connectivity to ensure that any user can still get into their own home.
    \newline
\end{itemize}

    
PIN codes can be assigned, revoked, or updated remotely by authorized users through the cloud interface. After multiple failed attempts, the lock will notify the user and enter a temporary lockout state to prevent brute-force attacks. The cloud and app logs all PIN code activities, including successful and failed attempts.

\subsection{WiFi Connection}
Smartlocks connect to the internet via WiFi, through a WiFi accesspoint model. This allows for the user to smoothly connect their smartlock to their home network.

\subsubsection*{Smartlock WiFi Features}
\begin{itemize}
    \item SmartLock needs WiFi in order to initially connect to the cloud.
    \item WiFi SSID and password are not stored in the lock, cloud, or app ensuring security.
    \item After connecting smartlock to WiFi through accesspoint mode and entering your smartlock's personal API key, the lock will be operational.
\end{itemize}

The WiFi module is configured during device setup and supports secure protocols (e.g., WPA2) to protect network traffic. If a user want's to change their WiFi network, they can do so through the same process as the initial setup. The lock will enter accesspoint mode, and the user can connect to the lock's WiFi network to change the SSID and password.

\subsection{Many-to-Many Relationship: Locks and Users}
The system supports a many-to-many relationship between users and locks:
\begin{itemize}
    \item Each user can have access to multiple locks
    \item Each lock can be accessed by multiple users
    \item Access rights are managed in the cloud database, with mapping tables linking users and locks
    \item Permissions can be updated dynamically, supporting temporary or permanent access
\end{itemize}
This flexible model allows for efficient management of access in environments such as offices, apartments, or shared facilities.

\subsection{Conclusion}
The Smartlock system integrates cloud services, secure authentication, wireless connectivity, and a flexible data model to deliver a robust and scalable access control solution.

\end{document}