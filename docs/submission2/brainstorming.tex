\subsection{Brainstorming}

\subsubsection{Adam Wu}
\begin{itemize}
    \item As a person who has amnesia, I would like to be able to find my keys anytime so that when I forget where I place them, I can find them.
    \begin{itemize}
        \item Having a "find my" solution with a key.
    \end{itemize}
    \item As a person who loves security, I would like to have the best lock for my house so that lock pickers are not able to pick my lock.
    \begin{itemize}
        \item Making an “authentication” key that resets the key code within a set time, making it harder for hackers to unlock the door.
    \end{itemize}
    \item As a person who is always last-minute out the door, I fear forgetting to lock the door when I close it.
    \begin{itemize}
        \item Auto-locking door when a person closes the door.
    \end{itemize}
    \item As a person who often forgets to bring their keys, I am scared of getting locked out.
    \begin{itemize}
        \item Having a notification from the key to the phone that alerts: “keys are not close by to you.”
    \end{itemize}
    \item As a parent, I am scared of my kids forgetting their keys and locking themselves out of their room.
    \begin{itemize}
        \item Creating a “master key” that only parents/admins can use to unlock specific doors.
    \end{itemize}
    \item Concerned about key battery life.
    \begin{itemize}
        \item Send a notification to the user when the key is on low battery.
    \end{itemize}
\end{itemize}

\subsubsection{Nathaniel Laurente}
\begin{itemize}
    \item Key has the ability to notify the user when too far away from the user’s phone/body.
    \item Key deactivates/won’t be able to open the door if too far away from the owner.
    \item “Tap to Pay” technology concept.
    \begin{itemize}
        \item Unlocks the door like a credit card tap on a phone.
        \item If too complex, explore alternative ways to unlock the door.
        \item Eliminates the need for a physical key.
        \item Prevents stolen keys from working if the user still has their phone.
    \end{itemize}
    \item Secure deactivation of the key when too far from the user.
    \begin{itemize}
        \item Possible solution: Use the user's phone for deactivation.
    \end{itemize}
    \item One-time password generator between lock and key to ensure only this exact key can enter the house.
    \item Backup way to get into the house if the user forgets/loses their key.
    \begin{itemize}
        \item Pin access code.
        \item App allows for 2FA authentication using a thumbprint and/or Face ID.
    \end{itemize}
    \item Will the battery last long enough for multiple years?
\end{itemize}

\subsubsection{Neena Nguyen}
\begin{itemize}
    \item Existing smart lock solutions:
    \begin{itemize}
        \item Smart locks for dorm rooms using mobile apps, passcodes, and scanners.
    \end{itemize}
    \item Who will use this lock?
    \begin{itemize}
        \item People with memory issues (elderly, ADHD).
        \item University students in dorm rooms.
        \item Student ID scanner integration.
        \item Parents with small children (child-proof locks).
    \end{itemize}
    \item Features for parental control.
    \begin{itemize}
        \item Locks after a curfew time.
        \item Prevents children from unlocking without parental approval.
        \item Alerts parents when kids come home from school.
    \end{itemize}
    \item What kind of door lock will it be?
    \begin{itemize}
        \item Facial recognition (requires camera and database knowledge).
        \item Logs entry and exit timestamps.
        \item Digital passcode through an app.
        \item Auto-relocking mechanism after failed attempts.
        \item Bluetooth detection for unlocking within a certain range.
        \item Dual authentication (PIN + scan).
        \item Optional security trigger after specific hours.
        \item Alerts when the door is left unlocked for too long.
        \item Auto-locking after prolonged unlocking.
        \item Detection system to check if the key is on the person.
        \item Prevents intruders from entering without a key.
    \end{itemize}
\end{itemize}

\subsubsection{Jackson Kennedy}
\begin{itemize}
    \item Normal keys can be lock-picked, but digital keys can be secured based on a communication protocol.
    \item Secure authentication methods.
    \begin{itemize}
        \item PIN authentication with 2FA.
        \item Optimal PIN length (e.g., 4-digit PIN has 1,048,576 combinations).
        \item Brute force prevention strategies.
    \end{itemize}
    \item Preventing communication protocol vulnerabilities.
    \begin{itemize}
        \item What protocol should be used? (Bluetooth has vulnerabilities and short range.)
        \item Cloud-based solutions rely on third-party vendor security.
        \item What information needs to be transferred? (Video data, authentication signals?)
    \end{itemize}
    \item Lock activation logic.
    \begin{itemize}
        \item How exactly will the lock know when to unlock? (Sending a 0 or 1 signal based on specific conditions?)
    \end{itemize}
    \item Security and alerting technologies.
    \begin{itemize}
        \item Sensors to detect nearby people.
        \item Hidden camera or biometric verification for identity confirmation.
    \end{itemize}
    \item Scheduling and timed access.
    \begin{itemize}
        \item Physical locks do not have scheduling options.
        \item Implement timed unlocking (e.g., unlock for 15 minutes for a babysitter).
        \item Extra verification to prevent intruders from exploiting schedules.
    \end{itemize}
\end{itemize}
