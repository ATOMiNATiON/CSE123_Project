\documentclass[12pt]{article}
\usepackage{enumitem}
\usepackage{titlesec}
\titleformat{\section}{\large\bfseries}{\thesection.}{0.5em}{}
\titleformat{\subsection}{\normalsize\bfseries}{\thesubsection.}{0.5em}{}
\usepackage{graphicx}
\usepackage{subcaption}

% Margins
\topmargin=-0.45in
\evensidemargin=0in
\oddsidemargin=0in
\textwidth=6.5in
\textheight=9.0in
\headsep=0.25in

\title{ SmartLock Demo Script }
\author{ Adam Wu, Jackson Kennedy, Neena Nguyen, and Nathaniel Laurente }
\date{\today}

\begin{document}
\maketitle

\section*{Before the Demo}
\begin{itemize}[leftmargin=1.5em]
    \item Prepare all text messages in advance to copy and paste during the demo.
    \item All components should be powered by separate batteries, \textbf{not} the laptop.
    \item The lock must be plugged into a \textbf{12V battery}, not the laptop.
    \item The ESP32’s \textbf{5V input} should also be powered by a battery.
    \item Set up should look realistic — ideally, the lock should appear connected to a wall power source.
    \item Be very clear when demonstrating re-authentication steps.
\end{itemize}

\section*{Introduction}
\textbf{Neena:} 
“We built a smart lock solution that lets users lock and unlock their doors from anywhere, at any time—no physical key needed. Our lock supports secure authentication methods and provides flexible, time-based access control.”\\
\\
\textbf{Adam:} 
“Here’s our schematic: we have our ESP32-C3 microcontroller and iOS app connected through Firestore Cloud. We handle Wi-Fi configuring using a one time set up access point, receive keypad inputs, and manage OTPs and emergency pins through the app. We also implemented features like logout, access logs, and a secure database for PIN handling.”\\
\\
\textbf{Nathaniel:}
“For the prototype, we’re using a 12V power supply, but our actual product design will include a 12V rechargeable battery.”\\

\section{Scenario 1: Quick Remote Unlock}
\textit{Demonstrates quick remote unlock via app.}

\begin{enumerate}[leftmargin=1.5em]
    \item Neena: “Nathaniel is coming home and forgot his keys. Adam plays the dad who’s at work.”
    \item Nathaniel: “I forgot my keys, can you unlock the door for me?”
    \item Adam: “I just got the text and will press \textbf{Unlock} on the app.”
    \item Neena: “You can see the lock has retracted slightly. We’ll show it again for clarity—it’s a subtle visual change.”
    \item Nathaniel texts back: “I’m in.”
    \item Adam presses \textbf{Lock}.
    \item Repeat the unlock/lock flow to show reliability.
    \item Neena: “In the final version, we plan a notification system that alerts users if the door remains unlocked.”
\end{enumerate}

\section{Scenario 2: Guest One-Time PIN}
\textit{Demonstrates one-time PIN functionality for visitors without app access.}

